%Idioma español y acentos
\usepackage[spanish]{babel}
%\usepackage[latin1]{inputenc}
\usepackage[utf8]{inputenc}

%algunos sÌmbolos matemáticos y paquetes para usar subimágenes
\usepackage{amsmath}
\usepackage{amsfonts}
\usepackage{amssymb}
\usepackage{graphicx}
\usepackage{listings}
\usepackage{appendix}
%Márgenes
\usepackage[left=3cm,top=3cm,right=3cm,bottom=3cm]{geometry}

%multi columnas
\usepackage{multicol}

%colores
\usepackage[dvipsnames]{xcolor}

%código
\usepackage{verbatim}

%para generar índice con hipervínculos
\usepackage{hyperref}
\hypersetup{
    colorlinks=true,       % false: boxed links; true: colored links
    linkcolor=Brown,          % color of internal links
    citecolor=BrickRed,        % color of links to bibliography
    filecolor=black,      % color of file links
    urlcolor=Blue           % color of external links 
}

%nomenclatura
\usepackage{nomencl}
\makenomenclature

%traducciones
\addto\captionsspanish{%
	\def\tablename{Tabla}%
	\def\listtablename{\'Indice de tablas}}

\renewcommand\nomname{Nomenclatura}
\def\nompreamble{\addcontentsline{toc}{chapter}{\nomname}\markboth{\nomname}{\nomname}}
