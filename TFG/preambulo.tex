%Idioma español y acentos
\usepackage[spanish]{babel}
%\usepackage[latin1]{inputenc}
\usepackage[utf8]{inputenc}

%algunos sÌmbolos matemáticos y paquetes para usar subimágenes
\usepackage{amsmath}
\usepackage{amsfonts}
\usepackage{amssymb}
\usepackage{graphicx}
\usepackage{listings}
\usepackage{appendix}
%Márgenes
\usepackage[left=3cm,top=3cm,right=3cm,bottom=3cm]{geometry}

%multi columnas
\usepackage{multicol}

%colores
\usepackage[dvipsnames]{xcolor}

%código
\usepackage{verbatim}

%para generar índice con hipervínculos
\usepackage{hyperref}

%parametros del documento (sus propiedades)
\hypersetup{
	pdftitle={Francisco Delgado López - TFG - 2024},
	pdfsubject={TFG - 2024},
	pdfauthor={Francisco Delgado López},
	pdfkeywords={geometría solar, radiación solar, energía solar, fotovoltaica, métodos de visualización, series temporales, datos espacio-temporales, S4},
	colorlinks,
	citecolor=BrickRed,
	filecolor=black,
	linkcolor=Brown,
	urlcolor=Blue,
}

%nomenclatura
\usepackage{nomencl}
\makenomenclature

%traducciones
\addto\captionsspanish{%
	\def\tablename{Tabla}%
	\def\listtablename{\'Indice de tablas}}

\renewcommand\nomname{Nomenclatura}
\def\nompreamble{\addcontentsline{toc}{chapter}{\nomname}\markboth{\nomname}{\nomname}}
