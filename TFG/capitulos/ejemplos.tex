\chapter{Ejemplo práctico de aplicación}
\label{chap:ejemplo-practico-aplicacion}
Una vez explicado como funciona el paquete, se puede realizar una demostración práctica tomando como ejemplo los módulos fotovoltaicos que tiene en su azotea la Escuela Técnica Superior de Ingeniería y Diseño Industrial (en adelante la ETSIDI).

Se tomará de base un estudio realizado por profesores de la escuela \cite{adrada17}, en el cual, comparan la producción energética de seis tipos de tecnologías fotovoltaicas.

En este ejemplo se realizará el mismo análisis tomando tres herramientas distintas: \texttt{solaR}, para poder tomar como referencia el paquete del que sale para poder apreciar las mejoras del programa, \texttt{PVSyst}, ya que es uno de los softwares más usados en el ámbito de la fotovoltaica y puede servir como punto de referencia, y por último \texttt{solaR2}.

\section{\texttt{solaR}}
\label{sec:orgd052418}
Se empieza inicilizando el paquete:
\begin{lstlisting}[language=r,label= ,caption= ,captionpos=b,numbers=none]
library(solaR)
\end{lstlisting}

Leemos los datos extraidos del estudio mencionado previamente. 

\section{\texttt{PVsyst}}
\label{sec:orgda15b80}

\ldots{}


\section{\texttt{solaR2}}
\label{sec:org40adc72}

\ldots{}

\section{Comparación entre los tres}
\label{sec:orgaac1a84}
