\chapter{Ejemplo práctico de aplicación}
\label{chap:ejemplo-practico-aplicacion}
Una vez explicado como funciona el paquete, se puede realizar una demostración práctica tomando como ejemplo los módulos fotovoltaicos que tiene en su azotea la Escuela Técnica Superior de Ingeniería y Diseño Industrial (en adelante la ETSIDI).

Se tomará de base un estudio realizado por profesores de la escuela \cite{adrada17}, en el cual, comparan la producción energética de seis tipos de tecnologías fotovoltaicas.

En este ejemplo se realizará el mismo análisis tomando tres herramientas distintas: \texttt{solaR}, para poder tomar como referencia el paquete del que sale para poder apreciar las mejoras del programa, \texttt{PVSyst}, ya que es uno de los softwares más usados en el ámbito de la fotovoltaica y puede servir como punto de referencia, y por último \texttt{solaR2}.

\section{\texttt{solaR}}
\label{sec:org48f705f}
\label{sec:solaR}
Se empieza inicilizando el paquete:
\begin{lstlisting}[language=r,label= ,caption= ,captionpos=b,numbers=none]
library(solaR)
\end{lstlisting}

El estudio antes mencionado recopila medias mensuales de la irradiación global incidente en un ángulo de inclinación y azimutal de 30º y 19ºE, respectivamente. Tanto \texttt{solaR} como \texttt{solaR2} emplean datos de irradiación pero en el plano horizontal, por lo que se empleará una proporción para obtener estos datos.
\begin{lstlisting}[language=r,label= ,caption= ,captionpos=b,numbers=none]
### Datos del estudio
## Valores de la columna PIR_ave
Gef_pir <- c(102.5, 106.2, 150.0, 156.8, 211.6, 217.7,
             219.5, 224.1, 181.0, 124.4, 110.2, 96.3)

## Valores de la columna CRC_ave
Gef_crc <- c(102.7, 104.6, 147.0, 157.9, 203.2, 207.7,
             219.8, 214.7, 176.1, 126.5, 109.1, 98.1)

## Se obtiene la media de los valores en Wh/m2
Gef_estudio <- mapply(function(x, y) (x+y)/2,
                      Gef_pir, Gef_crc)
show(Gef_estudio)
\end{lstlisting}
Con estos datos se va a calcular un objeto \texttt{Gef} tomando como datos de irradiación horizontal los obtenidos anteriormente, de esta manera se puede obtener una relación entre irradiación incidente y horizonal.
\begin{lstlisting}[language=r,label= ,caption= ,captionpos=b,numbers=none]
gef_estudio <- calcGef(lat = 40.4, modeTrk = 'fixed', modeRad = 'prom',
                       dataRad = list(G0dm = Gef_estudio, year = 2013),
                       beta = 30, alfa = -19,
                       iS = 1, alb = 0.2)
gefdm <- as.data.frameM(gef_estudio, complete = TRUE)
gefgo <- with(gefdm, Gefd/G0d)
G0_estudio <- Gef_estudio*gefgo
G0_estudio
\end{lstlisting}

Una vez se tienen estos datos, se puede calcular la producción que van a tener los diferentes sistemas fotovoltaicos.

Para ello, se necesitan los parámetros de los diferentes sistemas. En la tabla \ref{tab:parametros-tecnicos-modulos-fotovoltaicos} se pueden ver los distintos parámetros de los módulos fotovoltaicos.
\begin{center}
{\scriptsize }%
\begin{table}[]
{\scriptsize \caption{Parámetros técnicos de diferentes tipos de células solares.\label{tab:parametros-tecnicos-modulos-fotovoltaicos}}}
\centering{}{\scriptsize }\begin{tabular}{>{\centering}m{4.2cm} *{6}{>{\centering}m{1.48cm}}}
\toprule 
{\scriptsize \textbf{Parámetros Técnicos}} & {\scriptsize \textbf{mc-Si}} & {\scriptsize \textbf{pc-Si}} & {\scriptsize \textbf{a-Si/$\mu$c-Si}} & {\scriptsize \textbf{CdTe/Cds}} & {\scriptsize \textbf{CIS}} & {\scriptsize \textbf{mc-dc-Si}}\tabularnewline
\midrule
{\scriptsize Potencia se salida (Wp)} & {\scriptsize 250} & {\scriptsize 220} & {\scriptsize 115} & {\scriptsize 77.5} & {\scriptsize 73} & {\scriptsize 333}\tabularnewline
{\scriptsize Voltaje en $P_{max}$ (Vmp)} & {\scriptsize 29.9} & {\scriptsize 29.0} & {\scriptsize 174} & {\scriptsize 46.7} & {\scriptsize 33.9} & {\scriptsize 54.7}\tabularnewline
{\scriptsize Corriente en $P_{max}$ (Imp)} & {\scriptsize 8.37} & {\scriptsize 7.59} & {\scriptsize 0.661} & {\scriptsize 1.68} & {\scriptsize 2.21} & {\scriptsize 6.09}\tabularnewline
{\scriptsize Voltaje en circuito abierto (Voc)} & {\scriptsize 37.1} & {\scriptsize 36.5} & {\scriptsize 238} & {\scriptsize 62.5} & {\scriptsize 43.1} & {\scriptsize 65.3}\tabularnewline
{\scriptsize Corriente en cortocircuito (Isc)} & {\scriptsize 8.76} & {\scriptsize 8.15} & {\scriptsize 0.810} & {\scriptsize 1.98} & {\scriptsize 2.40} & {\scriptsize 6.46}\tabularnewline
{\scriptsize Eficiencia del módulo (\%)} & {\scriptsize 15.5} & {\scriptsize 14.4} & {\scriptsize 8.1} & {\scriptsize 10.4} & {\scriptsize 10.3} & {\scriptsize 20.4}\tabularnewline
{\scriptsize $\alpha_{Isc}$ (\%/K)} & {\scriptsize 0.0043} & {\scriptsize 0.06} & {\scriptsize 0.07} & {\scriptsize 0.02} & {\scriptsize 0.05} & {\scriptsize 0.054}\tabularnewline
{\scriptsize $\beta_{Voc}$ (\%/K)} & {\scriptsize -0.338} & {\scriptsize -0.37} & {\scriptsize -0.30} & {\scriptsize -0.24} & {\scriptsize -0.29} & {\scriptsize -0.27}\tabularnewline
{\scriptsize $\gamma_{Pmpp}$ (\%/K)} & {\scriptsize -0.469} & {\scriptsize -0.45} & {\scriptsize -0.24} & {\scriptsize -0.25} & {\scriptsize -0.36} & {\scriptsize -0.38}\tabularnewline
{\scriptsize Temperatura NOC (ºC)} & {\scriptsize 43.7} & {\scriptsize 46} & {\scriptsize 44} & {\scriptsize 40} & {\scriptsize 47} & {\scriptsize 45}\tabularnewline
\bottomrule
\end{tabular}
\end{table}
\end{center}
Se almacena esta información en listas con la información de cada módulo.

\begin{lstlisting}[language=r,label= ,caption= ,captionpos=b,numbers=none]
## mc-Si
module1 <- list(Vocn = 37.1,
                Iscn = 8.76,
                Vmn = 29.9,
                Imn = 8.37,
                CoefVT = 0.00338,
                TONC = 43.7)
## pc-Si
module2 <- list(Vocn = 36.5,
                Iscn = 8.15,
                Vmn = 29,
                Imn = 7.59,
                CoefVT = 0.0037,
                TONC = 46)
## a-Si/mc-Si
module3 <- list(Vocn = 238,
                Iscn = 0.81,
                Vmn = 174,
                Imn = 0.661,
                CoefVT = 0.003,
                TONC = 44)
## CdTe/Cds
module4 <- list(Vocn = 62.5,
                Iscn = 1.98,
                Vmn = 46.7,
                Imn = 1.68,
                CoefVT = 0.0024,
                TONC = 40)
## CIS
module5 <- list(Vocn = 43.1,
                Iscn = 2.4,
                Vmn = 33.9,
                Imn = 2.21,
                CoefVT = 0.0029,
                TONC = 47)
## mc-dc-Si
module6 <- list(Vocn = 65.3,
                Iscn = 6.46,
                Vmn = 54.7,
                Imn = 6.09,
                CoefVT = 0.0038,
                TONC = 45)
\end{lstlisting}

Una vez se tiene la información de cada tipo de módulo, en la tabla \ref{tab:sistemas-fotovoltaicos} se pueden ver la información de la agrupación de cada sistema.
\begin{center}
{\footnotesize }%
\begin{table}
{\scriptsize \caption{Sistemas fotovoltaicos.\label{tab:sistemas-fotovoltaicos}}}
\centering{}{\scriptsize }\begin{tabular}{*{7}{>{\centering}m{1.85cm}}}
\toprule 
{\scriptsize \textbf{Sistema}} & {\scriptsize \textbf{Tecnología}} & {\scriptsize \textbf{Año de Fabricación}} & {\scriptsize \textbf{Módulos en Serie}} & {\scriptsize \textbf{Módulos en Paralelo}} & {\scriptsize \textbf{Potencia del Sistema STC ($Wp_{STC}$)}} & {\scriptsize \textbf{Tamaño ($m^2$)}}\tabularnewline
\midrule
{\scriptsize 1} & {\scriptsize mc-Si} & {\scriptsize 2012} & {\scriptsize 5} & {\scriptsize 1} & {\scriptsize 1250} & {\scriptsize 8}\tabularnewline
{\scriptsize 2} & {\scriptsize pc-Si} & {\scriptsize 2009} & {\scriptsize 5} & {\scriptsize 1} & {\scriptsize 1100} & {\scriptsize 8.2}\tabularnewline
{\scriptsize 3} & {\scriptsize a-Si/$\mu$c-Si} & {\scriptsize 2009} & {\scriptsize 1} & {\scriptsize 10} & {\scriptsize 1150} & {\scriptsize 14.2}\tabularnewline
{\scriptsize 4} & {\scriptsize CdTe/Cds} & {\scriptsize 2010} & {\scriptsize 5} & {\scriptsize 2} & {\scriptsize 775} & {\scriptsize 7.2}\tabularnewline
{\scriptsize 5} & {\scriptsize CIS} & {\scriptsize 2008} & {\scriptsize 8} & {\scriptsize 2} & {\scriptsize 1175} & {\scriptsize 11.7}\tabularnewline
{\scriptsize 6} & {\scriptsize mc-dc-Si} & {\scriptsize 2012} & {\scriptsize 4} & {\scriptsize 1} & {\scriptsize 1332} & {\scriptsize 6.5}\tabularnewline
\bottomrule
\end{tabular}
\end{table}
\end{center}
De la misma manera, se almacenará esta información en listas.

\begin{lstlisting}[language=r,label= ,caption= ,captionpos=b,numbers=none]
## mc-Si
generator1 <- list(Nms = 5, Nmp = 1)
## pc-Si
generator2 <- list(Nms = 5, Nmp = 1)
## a-Si/mc-Si
generator3 <- list(Nms = 1, Nmp = 10)
## CdTe/Cds
generator4 <- list(Nms = 5, Nmp = 2)
## CIS
generator5 <- list(Nms = 8, Nmp = 2)
## mc-dc-Si
generator6 <- list(Nms = 4, Nmp = 1)
\end{lstlisting}

Una vez se tienen todos los parémtros del sistema fotovoltaico, se requieren los parámetros del inversor que tienen estos sistemas. Para facilitar el estudio, en el artículo explican que se usa el mismo inversor para todos los sistemas. Los parámetros de este se pueden ver en la tabla \ref{tab:caracteristicas-inversor}. 
\begin{center}
{\footnotesize }%
\begin{table}
{\scriptsize \caption{Carácteristicas del inversor.\label{tab:caracteristicas-inversor}}}
\centering{}{\scriptsize }\begin{tabular}{*{2}{>{\centering}m{5cm}}}
\toprule 
{\scriptsize \textbf{Inversor}} & {\scriptsize \textbf{SMA Sunny Boy-1200}} \tabularnewline
\midrule
{\scriptsize Potencia máxima DC} & {\scriptsize 1320 W} \tabularnewline
{\scriptsize Corriente máxima DC} & {\scriptsize 12.6 A} \tabularnewline
{\scriptsize Tensión máxima DC} & {\scriptsize 400 V} \tabularnewline
{\scriptsize Rango de tensión fotovoltaica (mpp)} & {\scriptsize 100-320 V} \tabularnewline
{\scriptsize Potencia máxima DC} & {\scriptsize 1320 W} \tabularnewline
{\scriptsize Potencia nominal de salida} & {\scriptsize 1200 W} \tabularnewline
{\scriptsize Maxima potencia aparente} & {\scriptsize 1200 VA} \tabularnewline
{\scriptsize Corriente máxima AC} & {\scriptsize 6.1 A}\tabularnewline
{\scriptsize Eficiencia} & {\scriptsize 92.1\%} \tabularnewline
\bottomrule
\end{tabular}
\end{table}
\end{center}

Se almacena esta información en otra lista:
\begin{lstlisting}[language=r,label= ,caption= ,captionpos=b,numbers=none]
inverter <- list(Pinv = 1200,
                 Vmin = 100,
                 Vmax = 320)
\end{lstlisting}

Una vez recopilada toda la información (la información que falta se deja sin añadir para que el propio paquete añada sus valores por defecto), se puede calcular la producción que tuvieron los sistemas:

\begin{lstlisting}[language=r,label= ,caption= ,captionpos=b,numbers=none]
prod1 <- prodGCPV(lat = 40.4, modeTrk = 'fixed', modeRad = 'prom',
                  dataRad = list(G0dm = G0_estudio, year = 2013),
                  beta = 30, alfa = -19, iS = 1,
                  module = module1, generator = generator1,
                  inverter = inverter)
show(prod1)
\end{lstlisting}
\begin{lstlisting}[language=r,label= ,caption= ,captionpos=b,numbers=none]
prod2 <- prodGCPV(lat = 40.4, modeTrk = 'fixed', modeRad = 'prom',
                  dataRad = list(G0dm = G0_estudio, year = 2013),
                  beta = 30, alfa = -19, iS = 1,
                  module = module2, generator = generator2,
                  inverter = inverter)
show(prod2)
\end{lstlisting}
\begin{lstlisting}[language=r,label= ,caption= ,captionpos=b,numbers=none]
prod3 <- prodGCPV(lat = 40.4, modeTrk = 'fixed', modeRad = 'prom',
                  dataRad = list(G0dm = G0_estudio, year = 2013),
                  beta = 30, alfa = -19, iS = 1,
                  module = module3, generator = generator3,
                  inverter = inverter)
show(prod3)
\end{lstlisting}
\begin{lstlisting}[language=r,label= ,caption= ,captionpos=b,numbers=none]
prod4 <- prodGCPV(lat = 40.4, modeTrk = 'fixed', modeRad = 'prom',
                  dataRad = list(G0dm = G0_estudio, year = 2013),
                  beta = 30, alfa = -19, iS = 1,
                  module = module4, generator = generator4,
                  inverter = inverter)
show(prod4)
\end{lstlisting}
\begin{lstlisting}[language=r,label= ,caption= ,captionpos=b,numbers=none]
prod5 <- prodGCPV(lat = 40.4, modeTrk = 'fixed', modeRad = 'prom',
                  dataRad = list(G0dm = G0_estudio, year = 2013),
                  beta = 30, alfa = -19, iS = 1,
                  module = module5, generator = generator5,
                  inverter = inverter)
show(prod1)
\end{lstlisting}
\begin{lstlisting}[language=r,label= ,caption= ,captionpos=b,numbers=none]
prod6 <- prodGCPV(lat = 40.4, modeTrk = 'fixed', modeRad = 'prom',
                  dataRad = list(G0dm = G0_estudio, year = 2013),
                  beta = 30, alfa = -19, iS = 1,
                  module = module6, generator = generator6,
                  inverter = inverter)
show(prod6)
\end{lstlisting}
\section{\texttt{PVsyst}}
\label{sec:orgdedd277}

\ldots{}


\section{\texttt{solaR2}}
\label{sec:org92d3b35}
\label{sec:solaR2}
Con los datos obtenidos en la sección \ref{sec:solaR}, hacemos la misma operación pero con el paquete \texttt{solaR2}.
\begin{lstlisting}[language=r,label= ,caption= ,captionpos=b,numbers=none]
library(solaR2)
\end{lstlisting}
\begin{lstlisting}[language=r,label= ,caption= ,captionpos=b,numbers=none]
prod1 <- prodGCPV(lat = 40.4, modeTrk = 'fixed', modeRad = 'prom',
                  dataRad = list(G0dm = G0_estudio, year = 2013),
                  beta = 30, alfa = -19, iS = 1,
                  module = module1, generator = generator1,
                  inverter = inverter)
show(prod1)
\end{lstlisting}
\begin{lstlisting}[language=r,label= ,caption= ,captionpos=b,numbers=none]
prod2 <- prodGCPV(lat = 40.4, modeTrk = 'fixed', modeRad = 'prom',
                  dataRad = list(G0dm = G0_estudio, year = 2013),
                  beta = 30, alfa = -19, iS = 1,
                  module = module2, generator = generator2,
                  inverter = inverter)
show(prod2)
\end{lstlisting}
\begin{lstlisting}[language=r,label= ,caption= ,captionpos=b,numbers=none]
prod3 <- prodGCPV(lat = 40.4, modeTrk = 'fixed', modeRad = 'prom',
                  dataRad = list(G0dm = G0_estudio, year = 2013),
                  beta = 30, alfa = -19, iS = 1,
                  module = module3, generator = generator3,
                  inverter = inverter)
show(prod3)
\end{lstlisting}
\begin{lstlisting}[language=r,label= ,caption= ,captionpos=b,numbers=none]
prod4 <- prodGCPV(lat = 40.4, modeTrk = 'fixed', modeRad = 'prom',
                  dataRad = list(G0dm = G0_estudio, year = 2013),
                  beta = 30, alfa = -19, iS = 1,
                  module = module4, generator = generator4,
                  inverter = inverter)
show(prod4)
\end{lstlisting}
\begin{lstlisting}[language=r,label= ,caption= ,captionpos=b,numbers=none]
prod5 <- prodGCPV(lat = 40.4, modeTrk = 'fixed', modeRad = 'prom',
                  dataRad = list(G0dm = G0_estudio, year = 2013),
                  beta = 30, alfa = -19, iS = 1,
                  module = module5, generator = generator5,
                  inverter = inverter)
show(prod1)
\end{lstlisting}
\begin{lstlisting}[language=r,label= ,caption= ,captionpos=b,numbers=none]
prod6 <- prodGCPV(lat = 40.4, modeTrk = 'fixed', modeRad = 'prom',
                  dataRad = list(G0dm = G0_estudio, year = 2013),
                  beta = 30, alfa = -19, iS = 1,
                  module = module6, generator = generator6,
                  inverter = inverter)
show(prod6)
\end{lstlisting}

\section{Comparación entre los tres}
\label{sec:org9cda98f}
