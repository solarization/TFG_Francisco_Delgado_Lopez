\chapter{Ejemplo práctico de aplicación}
\label{chap:ejemplo-practico-aplicacion}
Una vez explicado como funciona el paquete, se puede realizar una demostración práctica tomando como ejemplo los módulos fotovoltaicos que tiene en su azotea la Escuela Técnica Superior de Ingeniería y Diseño Industrial (en adelante la ETSIDI).

Se tomará de base un estudio realizado por profesores de la escuela \cite{adrada17}, en el cual, comparan la producción energética de seis tipos de tecnologías fotovoltaicas.

En este ejemplo se realizará el mismo análisis tomando tres herramientas distintas: \texttt{solaR}, para poder tomar como referencia el paquete del que sale para poder apreciar las mejoras del programa, \texttt{PVSyst}, ya que es uno de los softwares más usados en el ámbito de la fotovoltaica y puede servir como punto de referencia, y por último \texttt{solaR2}.

\section{\texttt{solaR}}
\label{sec:org28bb0be}
\label{sec:solaR}
Se empieza inicilizando el paquete:
\begin{lstlisting}[numbers=left,language=r,label= ,caption= ,captionpos=b]
library(solaR)
\end{lstlisting}

\begin{verbatim}
Cargando paquete requerido: zoo

Adjuntando el paquete: 'zoo'

The following objects are masked from 'package:base':

    as.Date, as.Date.numeric

Cargando paquete requerido: lattice
Cargando paquete requerido: latticeExtra
Time Zone set to UTC.
\end{verbatim}

En el estudio anterior, se recopilaron datos intradiarios de irradiación los cuales fueron almacenados en archivos.
\begin{lstlisting}[numbers=left,language=r,label= ,caption= ,captionpos=b]
enemar13 <- readBDi(file = 'TFG/data/ETSIDI/etsidi/EneMar2013_1.csv',
                    lat = 40.4, time.col = 'Fecha')
show(enemar13)
\end{lstlisting}

\begin{verbatim}
Object of class  Meteo 

Source of meteorological information: bdI-TFG/data/ETSIDI/etsidi/EneMar2013_1.csv 
Latitude of source:  40.4 degrees

Meteorological Data:
     Index                              G0              Ta        
 Min.   :2013-01-24 00:15:00.00   Min.   :  0.0   Min.   :-33.80  
 1st Qu.:2013-02-09 18:00:00.00   1st Qu.: 13.0   1st Qu.:  9.50  
 Median :2013-02-26 11:15:00.00   Median : 13.0   Median : 12.20  
 Mean   :2013-02-26 11:19:20.84   Mean   :113.2   Mean   : 11.97  
 3rd Qu.:2013-03-15 04:30:00.00   3rd Qu.:135.0   3rd Qu.: 14.40  
 Max.   :2013-03-31 23:45:00.00   Max.   :755.0   Max.   : 64.50
\end{verbatim}

Una vez se tienen estos datos, se puede calcular la producción que van a tener los diferentes sistemas fotovoltaicos.

Para ello, se necesitan los parámetros de los diferentes sistemas. En la tabla \ref{tab:parametros-tecnicos-modulos-fotovoltaicos} se pueden ver los distintos parámetros de los módulos fotovoltaicos.
\begin{center}
{\scriptsize }%
\begin{table}[]
{\scriptsize \caption{Parámetros técnicos de diferentes tipos de células solares.\label{tab:parametros-tecnicos-modulos-fotovoltaicos}}}
\centering{}{\scriptsize }\begin{tabular}{>{\centering}m{5cm} *{2}{>{\centering}m{2cm}}}
\toprule 
{\scriptsize \textbf{Parámetros Técnicos}} & {\scriptsize \textbf{mc-Si}} & {\scriptsize \textbf{pc-Si}}\tabularnewline
\midrule
{\scriptsize Potencia se salida (Wp)} & {\scriptsize 250} & {\scriptsize 220}\tabularnewline
{\scriptsize Voltaje en $P_{max}$ (Vmp)} & {\scriptsize 29.9} & {\scriptsize 29.0}\tabularnewline
{\scriptsize Corriente en $P_{max}$ (Imp)} & {\scriptsize 8.37} & {\scriptsize 7.59}\tabularnewline
{\scriptsize Voltaje en circuito abierto (Voc)} & {\scriptsize 37.1} & {\scriptsize 36.5}\tabularnewline
{\scriptsize Corriente en cortocircuito (Isc)} & {\scriptsize 8.76} & {\scriptsize 8.15}\tabularnewline
{\scriptsize Eficiencia del módulo (\%)} & {\scriptsize 15.5} & {\scriptsize 14.4} \tabularnewline
{\scriptsize $\alpha_{Isc}$ (\%/K)} & {\scriptsize 0.0043} & {\scriptsize 0.06} \tabularnewline
{\scriptsize $\beta_{Voc}$ (\%/K)} & {\scriptsize -0.338} & {\scriptsize -0.37}\tabularnewline
{\scriptsize $\gamma_{Pmpp}$ (\%/K)} & {\scriptsize -0.469} & {\scriptsize -0.45}\tabularnewline
{\scriptsize Temperatura NOC (ºC)} & {\scriptsize 43.7} & {\scriptsize 46}\tabularnewline
\bottomrule
\end{tabular}
\end{table}
\end{center}
Se almacena esta información en listas con la información de cada módulo.

\begin{lstlisting}[numbers=left,language=r,label= ,caption= ,captionpos=b]
## mc-Si
module1 <- list(Vocn = 37.1,
                Iscn = 8.76,
                Vmn = 29.9,
                Imn = 8.37,
                Ncs = 60,
                Ncp = 1,
                CoefVT = 0.00338,
                TONC = 43.7)
## pc-Si
module2 <- list(Vocn = 36.5,
                Iscn = 8.15,
                Vmn = 29,
                Imn = 7.59,
                Ncs = 60,
                Ncp = 1,
                CoefVT = 0.0037,
                TONC = 46)
\end{lstlisting}

Una vez se tiene la información de cada tipo de módulo, en la tabla \ref{tab:sistemas-fotovoltaicos} se pueden ver la información de la agrupación de cada sistema.
\begin{center}
{\footnotesize }%
\begin{table}
{\scriptsize \caption{Sistemas fotovoltaicos.\label{tab:sistemas-fotovoltaicos}}}
\centering{}{\scriptsize }\begin{tabular}{*{7}{>{\centering}m{1.85cm}}}
\toprule 
{\scriptsize \textbf{Sistema}} & {\scriptsize \textbf{Tecnología}} & {\scriptsize \textbf{Año de Fabricación}} & {\scriptsize \textbf{Módulos en Serie}} & {\scriptsize \textbf{Módulos en Paralelo}} & {\scriptsize \textbf{Potencia del Sistema STC ($Wp_{STC}$)}} & {\scriptsize \textbf{Tamaño ($m^2$)}}\tabularnewline
\midrule
{\scriptsize 1} & {\scriptsize mc-Si} & {\scriptsize 2012} & {\scriptsize 5} & {\scriptsize 1} & {\scriptsize 1250} & {\scriptsize 8}\tabularnewline
{\scriptsize 2} & {\scriptsize pc-Si} & {\scriptsize 2009} & {\scriptsize 5} & {\scriptsize 1} & {\scriptsize 1100} & {\scriptsize 8.2}\tabularnewline
\bottomrule
\end{tabular}
\end{table}
\end{center}
De la misma manera, se almacenará esta información en listas.

\begin{lstlisting}[numbers=left,language=r,label= ,caption= ,captionpos=b]
## mc-Si
generator1 <- list(Nms = 5, Nmp = 1)
## pc-Si
generator2 <- list(Nms = 5, Nmp = 1)
\end{lstlisting}

Una vez se tienen todos los parémtros del sistema fotovoltaico, se requieren los parámetros del inversor que tienen estos sistemas. Para facilitar el estudio, en el artículo explican que se usa el mismo inversor para todos los sistemas. Los parámetros de este se pueden ver en la tabla \ref{tab:caracteristicas-inversor}. 
\begin{center}
{\footnotesize }%
\begin{table}
{\scriptsize \caption{Carácteristicas del inversor.\label{tab:caracteristicas-inversor}}}
\centering{}{\scriptsize }\begin{tabular}{*{2}{>{\centering}m{5cm}}}
\toprule 
{\scriptsize \textbf{Inversor}} & {\scriptsize \textbf{SMA Sunny Boy-1200}} \tabularnewline
\midrule
{\scriptsize Potencia máxima DC} & {\scriptsize 1320 W} \tabularnewline
{\scriptsize Corriente máxima DC} & {\scriptsize 12.6 A} \tabularnewline
{\scriptsize Tensión máxima DC} & {\scriptsize 400 V} \tabularnewline
{\scriptsize Rango de tensión fotovoltaica (mpp)} & {\scriptsize 100-320 V} \tabularnewline
{\scriptsize Potencia máxima DC} & {\scriptsize 1320 W} \tabularnewline
{\scriptsize Potencia nominal de salida} & {\scriptsize 1200 W} \tabularnewline
{\scriptsize Maxima potencia aparente} & {\scriptsize 1200 VA} \tabularnewline
{\scriptsize Corriente máxima AC} & {\scriptsize 6.1 A}\tabularnewline
{\scriptsize Eficiencia} & {\scriptsize 92.1\%} \tabularnewline
\bottomrule
\end{tabular}
\end{table}
\end{center}

Se almacena esta información en otra lista:
\begin{lstlisting}[numbers=left,language=r,label= ,caption= ,captionpos=b]
inverter <- list(Pinv = 1200,
                 Vmin = 100,
                 Vmax = 320)
\end{lstlisting}

Una vez recopilada toda la información (la información que falta se deja sin añadir para que el propio paquete añada sus valores por defecto), se puede calcular la producción que tuvieron los sistemas:

\begin{lstlisting}[numbers=left,language=r,label= ,caption= ,captionpos=b]
prod1 <- prodGCPV(lat = 40.4, modeTrk = 'fixed', modeRad = 'bdI',
                  dataRad = enemar13,
                  beta = 30, alfa = -19, iS = 1,
                  module = module1, generator = generator1,
                  inverter = inverter)
show(prod1)
\end{lstlisting}

\begin{verbatim}
Object of class  ProdGCPV 

Source of meteorological information: bdI-TFG/data/ETSIDI/etsidi/EneMar2013_1.csv 

Latitude of source:  40.4 degrees
Latitude for calculations:  40.4 degrees

Monthly averages:
               Eac      Edc       Yf
ene. 2013 2.288657 2.544214 1.829001
feb. 2013 2.912867 3.246235 2.327844
mar. 2013 2.642931 2.958194 2.112123

Yearly values:
          Eac      Edc       Yf
2013 181.8004 202.9523 145.2875
-----------------
Mode of tracking:  fixed 
    Inclination:  30 
    Orientation:  -19 
-----------------
Generator:
    Modules in series:  5 
    Modules in parallel:  1 
    Nominal power (kWp):  1.3
\end{verbatim}

\begin{lstlisting}[numbers=left,language=r,label= ,caption= ,captionpos=b]
prod2 <- prodGCPV(lat = 40.4, modeTrk = 'fixed', modeRad = 'bdI',
                  dataRad = enemar13,
                  beta = 30, alfa = -19, iS = 1,
                  module = module2, generator = generator2,
                  inverter = inverter)
show(prod2)
\end{lstlisting}

\begin{verbatim}
Object of class  ProdGCPV 

Source of meteorological information: bdI-TFG/data/ETSIDI/etsidi/EneMar2013_1.csv 

Latitude of source:  40.4 degrees
Latitude for calculations:  40.4 degrees

Monthly averages:
               Eac      Edc       Yf
ene. 2013 1.995563 2.219924 1.813242
feb. 2013 2.546910 2.840829 2.314216
mar. 2013 2.324995 2.608686 2.112576

Yearly values:
          Eac      Edc       Yf
2013 159.3528 178.1718 144.7938
-----------------
Mode of tracking:  fixed 
    Inclination:  30 
    Orientation:  -19 
-----------------
Generator:
    Modules in series:  5 
    Modules in parallel:  1 
    Nominal power (kWp):  1.1
\end{verbatim}

\section{\texttt{PVsyst}}
\label{sec:org803de5f}
Con la herramienta \texttt{PVsyst}, se ha generado un año promedio de datos de irradiación en la localización y con estos datos se han obtenido dos informes (uno por cada sistema).

Por comodidad, en este documento se van a extraer solo unas tablas con los resultados principales, sin embargo los informes completos están disponibles en el \href{https://github.com/solarization/TFG\_Francisco\_Delgado\_Lopez}{github} del documento.

En las tablas \ref{tab:pvsyst1} y \ref{tab:pvsyst2} se tienen los resultados de la simulación de los sistemas.
\begin{center}
{\footnotesize }%
\begin{table}
{\scriptsize \caption{Energía media mensual estimada por \texttt{PVSyst} en $KWh$ del sistema 1.\label{tab:pvsyst1}}}
\centering{}{\scriptsize }\begin{tabular}{*{13}{>{\centering}m{0.75cm}}}
\toprule 
{\scriptsize \textbf{Ene}} & {\scriptsize \textbf{Feb}} & {\scriptsize \textbf{Mar}} & {\scriptsize \textbf{Abr}} & {\scriptsize \textbf{May}} & {\scriptsize \textbf{Jun}} & {\scriptsize \textbf{Jul}} & {\scriptsize \textbf{Ago}} & {\scriptsize \textbf{Sep}} & {\scriptsize \textbf{Oct}} & {\scriptsize \textbf{Nov}} & {\scriptsize \textbf{Dic}} & {\scriptsize \textbf{Total}}\tabularnewline
\midrule
{\scriptsize 3,7} & {\scriptsize 4,0} & {\scriptsize 5,6} & {\scriptsize 5,3} & {\scriptsize 6,7} & {\scriptsize 6,7} & {\scriptsize 7,9} & {\scriptsize 7,2} & {\scriptsize 6,4} & {\scriptsize 4,8} & {\scriptsize 3,5} & {\scriptsize 3,6} & {\scriptsize 1941,1} \tabularnewline
\bottomrule
\end{tabular}
\end{table}
\end{center}
\begin{center}
{\footnotesize }%
\begin{table}
{\scriptsize \caption{Energía media mensual estimada por \texttt{PVSyst} en $KWh$ del sistema 2.\label{tab:pvsyst2}}}
\centering{}{\scriptsize }\begin{tabular}{*{13}{>{\centering}m{0.75cm}}}
\toprule 
{\scriptsize \textbf{Ene}} & {\scriptsize \textbf{Feb}} & {\scriptsize \textbf{Mar}} & {\scriptsize \textbf{Abr}} & {\scriptsize \textbf{May}} & {\scriptsize \textbf{Jun}} & {\scriptsize \textbf{Jul}} & {\scriptsize \textbf{Ago}} & {\scriptsize \textbf{Sep}} & {\scriptsize \textbf{Oct}} & {\scriptsize \textbf{Nov}} & {\scriptsize \textbf{Dic}} & {\scriptsize \textbf{Total}}\tabularnewline
\midrule
{\scriptsize 4,3} & {\scriptsize 4,6} & {\scriptsize 6,4} & {\scriptsize 6,1} & {\scriptsize 7,3} & {\scriptsize 7,3} & {\scriptsize 8,3} & {\scriptsize 7,7} & {\scriptsize 6,9} & {\scriptsize 5,4} & {\scriptsize 4,1} & {\scriptsize 4,4} & {\scriptsize 2213,7} \tabularnewline
\bottomrule
\end{tabular}
\end{table}
\end{center}



\section{\texttt{solaR2}}
\label{sec:org57c03f0}
\label{sec:solaR2}
Con los datos obtenidos en la sección \ref{sec:solaR}, hacemos la misma operación pero con el paquete \texttt{solaR2}.
\begin{lstlisting}[numbers=left,language=r,label= ,caption= ,captionpos=b]
library(solaR2)
\end{lstlisting}

\begin{verbatim}
Cargando paquete requerido: data.table
data.table 1.15.4 using 6 threads (see ?getDTthreads).  Latest news: r-datatable.com
Cargando paquete requerido: lattice
Cargando paquete requerido: latticeExtra
Time Zone set to UTC.
\end{verbatim}


Para ello importamos de la misma manera los datos de radiación.
\begin{lstlisting}[numbers=left,language=r,label= ,caption= ,captionpos=b]
enemar13 <- readBDi(file = 'TFG/data/ETSIDI/etsidi/EneMar2013_1.csv',
                    lat = 40.4, dates.col = 'Fecha')
show(enemar13)
\end{lstlisting}

\begin{verbatim}
Object of class  Meteo 

Source of meteorological information: bdI-TFG/data/ETSIDI/etsidi/EneMar2013_1.csv 
Latitude of source:  40.4 degrees

Meteorological Data:
     Dates                              G0              Ta        
 Min.   :2013-01-24 00:15:00.00   Min.   :  0.0   Min.   :-33.80  
 1st Qu.:2013-02-09 18:00:00.00   1st Qu.: 13.0   1st Qu.:  9.50  
 Median :2013-02-26 11:15:00.00   Median : 13.0   Median : 12.20  
 Mean   :2013-02-26 11:19:20.84   Mean   :113.2   Mean   : 11.97  
 3rd Qu.:2013-03-15 04:30:00.00   3rd Qu.:135.0   3rd Qu.: 14.40  
 Max.   :2013-03-31 23:45:00.00   Max.   :755.0   Max.   : 64.50
\end{verbatim}

Con estos datos se procede al cálculo de la producción (los datos de los componentes del sistema son los mismos que los realizados en la sección \ref{sec:solaR}).
\begin{lstlisting}[numbers=left,language=r,label= ,caption= ,captionpos=b]
prod1 <- prodGCPV(lat = 40.4, modeTrk = 'fixed', modeRad = 'bdI',
                  dataRad = enemar13,
                  beta = 30, alfa = -19, iS = 1,
                  module = module1, generator = generator1,
                  inverter = inverter)
show(prod1)
\end{lstlisting}

\begin{verbatim}
Object of class  ProdGCPV 

Source of meteorological information: bdI-TFG/data/ETSIDI/etsidi/EneMar2013_1.csv 

Latitude of source:  40.4 degrees
Latitude for calculations:  40.4 degrees

Monthly avarages:
       Dates      Eac      Edc       Yf
      <char>    <num>    <num>    <num>
1: Jan. 2013 2.288657 2.544214 1.829001
2: Feb. 2013 2.912867 3.246235 2.327844
3: Mar. 2013 2.642931 2.958194 2.112123

Yearly values:
   Dates      Eac      Edc       Yf
   <int>    <num>    <num>    <num>
1:  2013 181.8004 202.9523 145.2875
-----------------
Mode of tracking:  fixed 
    Inclination:  30 
    Orientation:  -19 
-----------------
Generator:
    Modules in series:  5 
    Modules in parallel:  1 
    Nominal power (kWp):  1.3
\end{verbatim}

\begin{lstlisting}[numbers=left,language=r,label= ,caption= ,captionpos=b]
prod2 <- prodGCPV(lat = 40.4, modeTrk = 'fixed', modeRad = 'bdI',
                  dataRad = enemar13,
                  beta = 30, alfa = -19, iS = 1,
                  module = module2, generator = generator2,
                  inverter = inverter)
show(prod2)
\end{lstlisting}

\begin{verbatim}
Object of class  ProdGCPV 

Source of meteorological information: bdI-TFG/data/ETSIDI/etsidi/EneMar2013_1.csv 

Latitude of source:  40.4 degrees
Latitude for calculations:  40.4 degrees

Monthly avarages:
       Dates      Eac      Edc       Yf
      <char>    <num>    <num>    <num>
1: Jan. 2013 1.995563 2.219924 1.813242
2: Feb. 2013 2.546910 2.840829 2.314216
3: Mar. 2013 2.324995 2.608686 2.112576

Yearly values:
   Dates      Eac      Edc       Yf
   <int>    <num>    <num>    <num>
1:  2013 159.3528 178.1718 144.7938
-----------------
Mode of tracking:  fixed 
    Inclination:  30 
    Orientation:  -19 
-----------------
Generator:
    Modules in series:  5 
    Modules in parallel:  1 
    Nominal power (kWp):  1.1
\end{verbatim}

\section{Comparación y conclusiones}
\label{sec:org2bbb619}
\label{sec:comparacion-conclusiones}
Como se puede observar en las secciones anteriores, tanto el paquete \texttt{solaR} como el paquete \texttt{solaR2} ofrecen los mismos resultados ya que toman las mismas referencias y estudios para realizar los cáculos. Sin embargo, el paquete \texttt{solaR2}, a parte de la corrección de algunos erores, presenta unas claras ventajas frente a su antecesor. Estas son:
\begin{itemize}
\item \textbf{Eficiencia}: al estar basado en \texttt{data.table}, el paquete gana eficiencia en operaciones complejas. Para mostrar esto vamos a utilizar el paquete \texttt{microbenchmark}.
\begin{lstlisting}[numbers=left,language=r,label= ,caption= ,captionpos=b]
## Con el paquete solaR
library(microbenchmark)
prodGCPVcustom <- function(){  
  prod1 <- prodGCPV(lat = 40.4, modeTrk = 'fixed', modeRad = 'bdI',
                    dataRad = enemar13, beta = 30, alfa =-19,
                    iS = 1, module = module1,
                    generator = generator1, inverter = inverter)
}
microbenchmark(prodGCPVcustom(), times = 50)
\end{lstlisting}

\begin{verbatim}
Unit: milliseconds
             expr      min       lq     mean   median       uq      max neval
 prodGCPVcustom() 529.5188 543.3399 554.9077 548.5941 558.3901 626.5719    50
\end{verbatim}


\begin{lstlisting}[numbers=left,language=r,label= ,caption= ,captionpos=b]
## Con el paquete solaR2
library(microbenchmark)
prodGCPVcustom <- function(){  
  prod1 <- prodGCPV(lat = 40.4, modeTrk = 'fixed', modeRad = 'bdI',
                    dataRad = enemar13, beta = 30, alfa =-19,
                    iS = 1, module = module1,
                    generator = generator1, inverter = inverter)
}
microbenchmark(prodGCPVcustom(), times = 50)
\end{lstlisting}

\begin{verbatim}
Unit: milliseconds
             expr      min       lq     mean   median       uq      max neval
 prodGCPVcustom() 333.9282 338.6178 345.8622 341.9009 345.6647 402.8548    50
\end{verbatim}


Aquí se puede ver que la eficiencia mejora. Sin embargo, suponiendo que en vez de un sistema fijo, tuvieramos un sistema de seguimiento de doble eje y quisieramos optener la mejor combinación de distancias, se podría utilizar la función \texttt{optimShd}, la cual al ser una tarea muy exigente se aprecia con más detalle las virtudes del paquete \texttt{solaR2} gracias al uso de \texttt{data.table}.
\begin{lstlisting}[numbers=left,language=r,label= ,caption= ,captionpos=b]
## Con el paquete solaR
struct2x <- list(W = 23.11, L = 9.8, Nrow = 2, Ncol = 3)
dist2x <- list(Lew = c(30, 45),Lns = c(20, 40))
optimShdcustom <- function(){  
  optim <- optimShd(lat = 40.4, modeTrk = 'two', modeRad = 'bdI',
                    dataRad = enemar13, beta = 30, alfa =-19,
                    iS = 1, module = module1,
                    generator = generator1, inverter = inverter,
                    modeShd = c('area', 'prom'),
                    distances = dist2x, struct = struct2x,
                    res = 5, prog = FALSE)
}
microbenchmark(optimShdcustom(), times = 20)
\end{lstlisting}

\begin{verbatim}
Unit: seconds
             expr     min       lq     mean   median      uq      max neval
 optimShdcustom() 6.30659 6.387046 6.447254 6.453184 6.48896 6.610802    20
\end{verbatim}


\begin{lstlisting}[numbers=left,language=r,label= ,caption= ,captionpos=b]
## Con el paquete solaR2
struct2x <- list(W = 23.11, L = 9.8, Nrow = 2, Ncol = 3)
dist2x <- list(Lew = c(30, 45),Lns = c(20, 40))
optimShdcustom <- function(){  
  optim <- optimShd(lat = 40.4, modeTrk = 'two', modeRad = 'bdI',
                    dataRad = enemar13, beta = 30, alfa =-19,
                    iS = 1, module = module1,
                    generator = generator1, inverter = inverter,
                    modeShd = c('area', 'prom'),
                    distances = dist2x, struct = struct2x,
                    res = 5, prog = FALSE)
}
microbenchmark(optimShdcustom(), times = 20)
\end{lstlisting}

\begin{verbatim}
Unit: seconds
             expr      min       lq     mean   median       uq      max neval
 optimShdcustom() 5.121113 5.154294 5.175936 5.171816 5.182229 5.249169    20
\end{verbatim}
\end{itemize}
