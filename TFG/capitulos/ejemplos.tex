\chapter{Ejemplo práctico de aplicación}
\label{chap:ejemplo-practico-aplicacion}
Una vez explicado como funciona el paquete, se puede realizar una demostración práctica tomando como ejemplo los módulos fotovoltaicos que tiene en su azotea la Escuela Técnica Superior de Ingeniería y Diseño Industrial (en adelante la ETSIDI).

Se tomará de base un estudio realizado por profesores de la escuela \cite{adrada17}, en el cual, comparan la producción energética de seis tipos de tecnologías fotovoltaicas.

En este ejemplo se realizará el mismo análisis tomando tres herramientas distintas: \texttt{solaR}, para poder tomar como referencia el paquete del que sale para poder apreciar las mejoras del programa, \texttt{PVSyst}, ya que es uno de los softwares más usados en el ámbito de la fotovoltaica y puede servir como punto de referencia, y por último \texttt{solaR2}.

\section{\texttt{solaR}}
\label{sec:org0f6ccb3}
\label{sec:solaR}
Se empieza inicilizando el paquete:
\begin{lstlisting}[numbers=left,language=r,label= ,caption= ,captionpos=b]
library(solaR)
\end{lstlisting}

El estudio antes mencionado recopila medias mensuales de la irradiación global incidente en un ángulo de inclinación y azimutal de 30º y 19ºE, respectivamente. Tanto \texttt{solaR} como \texttt{solaR2} emplean datos de irradiación pero en el plano horizontal, por lo que se empleará una proporción para obtener estos datos.
\begin{lstlisting}[numbers=left,language=r,label= ,caption= ,captionpos=b]
### Datos del estudio
## Valores de la columna PIR_ave
Gef_pir <- c(102.5, 106.2, 150.0, 156.8, 211.6, 217.7,
             219.5, 224.1, 181.0, 124.4, 110.2, 96.3)

## Valores de la columna CRC_ave
Gef_crc <- c(102.7, 104.6, 147.0, 157.9, 203.2, 207.7,
             219.8, 214.7, 176.1, 126.5, 109.1, 98.1)

## Se obtiene la media de los valores en Wh/m2
Gef_estudio <- mapply(function(x, y) (x+y)/2,
                      Gef_pir, Gef_crc)
show(Gef_estudio)
\end{lstlisting}

\begin{verbatim}
[1] 102.60 105.40 148.50 157.35 207.40 212.70 219.65 219.40 178.55 125.45 109.65  97.20
\end{verbatim}


Con estos datos se va a calcular un objeto \texttt{Gef} tomando como datos de irradiación horizontal los obtenidos anteriormente, de esta manera se puede obtener una relación entre irradiación incidente y horizonal.
\begin{lstlisting}[numbers=left,language=r,label= ,caption= ,captionpos=b]
gef_estudio <- calcGef(lat = 40.4, modeTrk = 'fixed', modeRad = 'prom',
                       dataRad = list(G0dm = Gef_estudio, year = 2013),
                       beta = 30, alfa = -19,
                       iS = 1, alb = 0.2)
gefdm <- as.data.frameM(gef_estudio, complete = TRUE)
gefgo <- with(gefdm, Gefd/G0d)
G0_estudio <- Gef_estudio*gefgo
G0_estudio
\end{lstlisting}

\begin{verbatim}
 [1]  94.66548  96.09844 134.79220 142.08586 187.03553 191.60740 198.01539 198.35401 162.03355
[10] 114.38822 101.08676  90.16601
\end{verbatim}


Una vez se tienen estos datos, se puede calcular la producción que van a tener los diferentes sistemas fotovoltaicos.

Para ello, se necesitan los parámetros de los diferentes sistemas. En la tabla \ref{tab:parametros-tecnicos-modulos-fotovoltaicos} se pueden ver los distintos parámetros de los módulos fotovoltaicos.
\begin{center}
{\scriptsize }%
\begin{table}[]
{\scriptsize \caption{Parámetros técnicos de diferentes tipos de células solares.\label{tab:parametros-tecnicos-modulos-fotovoltaicos}}}
\centering{}{\scriptsize }\begin{tabular}{>{\centering}m{5cm} *{2}{>{\centering}m{2cm}}}
\toprule 
{\scriptsize \textbf{Parámetros Técnicos}} & {\scriptsize \textbf{mc-Si}} & {\scriptsize \textbf{pc-Si}}\tabularnewline
\midrule
{\scriptsize Potencia se salida (Wp)} & {\scriptsize 250} & {\scriptsize 220}\tabularnewline
{\scriptsize Voltaje en $P_{max}$ (Vmp)} & {\scriptsize 29.9} & {\scriptsize 29.0}\tabularnewline
{\scriptsize Corriente en $P_{max}$ (Imp)} & {\scriptsize 8.37} & {\scriptsize 7.59}\tabularnewline
{\scriptsize Voltaje en circuito abierto (Voc)} & {\scriptsize 37.1} & {\scriptsize 36.5}\tabularnewline
{\scriptsize Corriente en cortocircuito (Isc)} & {\scriptsize 8.76} & {\scriptsize 8.15}\tabularnewline
{\scriptsize Eficiencia del módulo (\%)} & {\scriptsize 15.5} & {\scriptsize 14.4} \tabularnewline
{\scriptsize $\alpha_{Isc}$ (\%/K)} & {\scriptsize 0.0043} & {\scriptsize 0.06} \tabularnewline
{\scriptsize $\beta_{Voc}$ (\%/K)} & {\scriptsize -0.338} & {\scriptsize -0.37}\tabularnewline
{\scriptsize $\gamma_{Pmpp}$ (\%/K)} & {\scriptsize -0.469} & {\scriptsize -0.45}\tabularnewline
{\scriptsize Temperatura NOC (ºC)} & {\scriptsize 43.7} & {\scriptsize 46}\tabularnewline
\bottomrule
\end{tabular}
\end{table}
\end{center}
Se almacena esta información en listas con la información de cada módulo.

\begin{lstlisting}[numbers=left,language=r,label= ,caption= ,captionpos=b]
## mc-Si
module1 <- list(Vocn = 37.1,
                Iscn = 8.76,
                Vmn = 29.9,
                Imn = 8.37,
                CoefVT = 0.00338,
                TONC = 43.7)
## pc-Si
module2 <- list(Vocn = 36.5,
                Iscn = 8.15,
                Vmn = 29,
                Imn = 7.59,
                CoefVT = 0.0037,
                TONC = 46)
\end{lstlisting}

Una vez se tiene la información de cada tipo de módulo, en la tabla \ref{tab:sistemas-fotovoltaicos} se pueden ver la información de la agrupación de cada sistema.
\begin{center}
{\footnotesize }%
\begin{table}
{\scriptsize \caption{Sistemas fotovoltaicos.\label{tab:sistemas-fotovoltaicos}}}
\centering{}{\scriptsize }\begin{tabular}{*{7}{>{\centering}m{1.85cm}}}
\toprule 
{\scriptsize \textbf{Sistema}} & {\scriptsize \textbf{Tecnología}} & {\scriptsize \textbf{Año de Fabricación}} & {\scriptsize \textbf{Módulos en Serie}} & {\scriptsize \textbf{Módulos en Paralelo}} & {\scriptsize \textbf{Potencia del Sistema STC ($Wp_{STC}$)}} & {\scriptsize \textbf{Tamaño ($m^2$)}}\tabularnewline
\midrule
{\scriptsize 1} & {\scriptsize mc-Si} & {\scriptsize 2012} & {\scriptsize 5} & {\scriptsize 1} & {\scriptsize 1250} & {\scriptsize 8}\tabularnewline
{\scriptsize 2} & {\scriptsize pc-Si} & {\scriptsize 2009} & {\scriptsize 5} & {\scriptsize 1} & {\scriptsize 1100} & {\scriptsize 8.2}\tabularnewline
\bottomrule
\end{tabular}
\end{table}
\end{center}
De la misma manera, se almacenará esta información en listas.

\begin{lstlisting}[numbers=left,language=r,label= ,caption= ,captionpos=b]
## mc-Si
generator1 <- list(Nms = 5, Nmp = 1)
## pc-Si
generator2 <- list(Nms = 5, Nmp = 1)
\end{lstlisting}

Una vez se tienen todos los parémtros del sistema fotovoltaico, se requieren los parámetros del inversor que tienen estos sistemas. Para facilitar el estudio, en el artículo explican que se usa el mismo inversor para todos los sistemas. Los parámetros de este se pueden ver en la tabla \ref{tab:caracteristicas-inversor}. 
\begin{center}
{\footnotesize }%
\begin{table}
{\scriptsize \caption{Carácteristicas del inversor.\label{tab:caracteristicas-inversor}}}
\centering{}{\scriptsize }\begin{tabular}{*{2}{>{\centering}m{5cm}}}
\toprule 
{\scriptsize \textbf{Inversor}} & {\scriptsize \textbf{SMA Sunny Boy-1200}} \tabularnewline
\midrule
{\scriptsize Potencia máxima DC} & {\scriptsize 1320 W} \tabularnewline
{\scriptsize Corriente máxima DC} & {\scriptsize 12.6 A} \tabularnewline
{\scriptsize Tensión máxima DC} & {\scriptsize 400 V} \tabularnewline
{\scriptsize Rango de tensión fotovoltaica (mpp)} & {\scriptsize 100-320 V} \tabularnewline
{\scriptsize Potencia máxima DC} & {\scriptsize 1320 W} \tabularnewline
{\scriptsize Potencia nominal de salida} & {\scriptsize 1200 W} \tabularnewline
{\scriptsize Maxima potencia aparente} & {\scriptsize 1200 VA} \tabularnewline
{\scriptsize Corriente máxima AC} & {\scriptsize 6.1 A}\tabularnewline
{\scriptsize Eficiencia} & {\scriptsize 92.1\%} \tabularnewline
\bottomrule
\end{tabular}
\end{table}
\end{center}

Se almacena esta información en otra lista:
\begin{lstlisting}[numbers=left,language=r,label= ,caption= ,captionpos=b]
inverter <- list(Pinv = 1200,
                 Vmin = 100,
                 Vmax = 320)
\end{lstlisting}

Una vez recopilada toda la información (la información que falta se deja sin añadir para que el propio paquete añada sus valores por defecto), se puede calcular la producción que tuvieron los sistemas:

\begin{lstlisting}[numbers=left,language=r,label= ,caption= ,captionpos=b]
prod1 <- prodGCPV(lat = 40.4, modeTrk = 'fixed', modeRad = 'prom',
                  dataRad = list(G0dm = G0_estudio, year = 2013),
                  beta = 30, alfa = -19, iS = 1,
                  module = module1, generator = generator1,
                  inverter = inverter)
show(prod1)
\end{lstlisting}

\begin{verbatim}
Object of class  ProdGCPV 

Source of meteorological information: prom- 

Latitude of source:  40.4 degrees
Latitude for calculations:  40.4 degrees

Monthly averages:
                  Eac        Edc         Yf
ene. 2013 0.000000000 0.00000000 0.00000000
feb. 2013 0.000000000 0.00000000 0.00000000
mar. 2013 0.000000000 0.00000000 0.00000000
abr. 2013 0.000000000 0.00000000 0.00000000
may. 2013 0.000000000 0.00000000 0.00000000
jun. 2013 0.000000000 0.00000000 0.00000000
jul. 2013 0.009613741 0.02216007 0.00768291
ago. 2013 0.032013331 0.06983410 0.02558375
sep. 2013 0.000000000 0.00000000 0.00000000
oct. 2013 0.000000000 0.00000000 0.00000000
nov. 2013 0.000000000 0.00000000 0.00000000
dic. 2013 0.000000000 0.00000000 0.00000000

Yearly values:
          Eac      Edc       Yf
2013 1.290439 2.851819 1.031266
-----------------
Mode of tracking:  fixed 
    Inclination:  30 
    Orientation:  -19 
-----------------
Generator:
    Modules in series:  5 
    Modules in parallel:  1 
    Nominal power (kWp):  1.3
\end{verbatim}

\begin{lstlisting}[numbers=left,language=r,label= ,caption= ,captionpos=b]
prod2 <- prodGCPV(lat = 40.4, modeTrk = 'fixed', modeRad = 'prom',
                  dataRad = list(G0dm = G0_estudio, year = 2013),
                  beta = 30, alfa = -19, iS = 1,
                  module = module2, generator = generator2,
                  inverter = inverter)
show(prod2)
\end{lstlisting}

\begin{verbatim}
Object of class  ProdGCPV 

Source of meteorological information: prom- 

Latitude of source:  40.4 degrees
Latitude for calculations:  40.4 degrees

Monthly averages:
                  Eac        Edc          Yf
ene. 2013 0.000000000 0.00000000 0.000000000
feb. 2013 0.000000000 0.00000000 0.000000000
mar. 2013 0.000000000 0.00000000 0.000000000
abr. 2013 0.000000000 0.00000000 0.000000000
may. 2013 0.000000000 0.00000000 0.000000000
jun. 2013 0.000000000 0.00000000 0.000000000
jul. 2013 0.007712197 0.02014982 0.007007584
ago. 2013 0.026013451 0.06349067 0.023636773
sep. 2013 0.000000000 0.00000000 0.000000000
oct. 2013 0.000000000 0.00000000 0.000000000
nov. 2013 0.000000000 0.00000000 0.000000000
dic. 2013 0.000000000 0.00000000 0.000000000

Yearly values:
          Eac      Edc        Yf
2013 1.045495 2.592855 0.9499751
-----------------
Mode of tracking:  fixed 
    Inclination:  30 
    Orientation:  -19 
-----------------
Generator:
    Modules in series:  5 
    Modules in parallel:  1 
    Nominal power (kWp):  1.1
\end{verbatim}

\section{\texttt{PVsyst}}
\label{sec:org2fbc53f}

\ldots{}


\section{\texttt{solaR2}}
\label{sec:orga52f088}
\label{sec:solaR2}
Con los datos obtenidos en la sección \ref{sec:solaR}, hacemos la misma operación pero con el paquete \texttt{solaR2}.
\begin{lstlisting}[numbers=left,language=r,label= ,caption= ,captionpos=b]
library(solaR2)
\end{lstlisting}

\begin{lstlisting}[numbers=left,language=r,label= ,caption= ,captionpos=b]
prod1 <- prodGCPV(lat = 40.4, modeTrk = 'fixed', modeRad = 'prom',
                  dataRad = list(G0dm = G0_estudio, year = 2013),
                  beta = 30, alfa = -19, iS = 1,
                  module = module1, generator = generator1,
                  inverter = inverter)
show(prod1)
\end{lstlisting}

\begin{verbatim}
Object of class  ProdGCPV 

Source of meteorological information: prom- 

Latitude of source:  40.4 degrees
Latitude for calculations:  40.4 degrees

Monthly avarages:
        Dates         Eac        Edc         Yf
       <char>       <num>      <num>      <num>
 1: Jan. 2013 0.000000000 0.00000000 0.00000000
 2: Feb. 2013 0.000000000 0.00000000 0.00000000
 3: Mar. 2013 0.000000000 0.00000000 0.00000000
 4: Apr. 2013 0.000000000 0.00000000 0.00000000
 5: May. 2013 0.000000000 0.00000000 0.00000000
 6: Jun. 2013 0.000000000 0.00000000 0.00000000
 7: Jul. 2013 0.009613741 0.02216007 0.00768291
 8: Aug. 2013 0.032013331 0.06983410 0.02558375
 9: Sep. 2013 0.000000000 0.00000000 0.00000000
10: Oct. 2013 0.000000000 0.00000000 0.00000000
11: Nov. 2013 0.000000000 0.00000000 0.00000000
12: Dec. 2013 0.000000000 0.00000000 0.00000000

Yearly values:
   Dates      Eac      Edc       Yf
   <int>    <num>    <num>    <num>
1:  2013 1.290439 2.851819 1.031266
-----------------
Mode of tracking:  fixed 
    Inclination:  30 
    Orientation:  -19 
-----------------
Generator:
    Modules in series:  5 
    Modules in parallel:  1 
    Nominal power (kWp):  1.3
\end{verbatim}

\begin{lstlisting}[numbers=left,language=r,label= ,caption= ,captionpos=b]
prod2 <- prodGCPV(lat = 40.4, modeTrk = 'fixed', modeRad = 'prom',
                  dataRad = list(G0dm = G0_estudio, year = 2013),
                  beta = 30, alfa = -19, iS = 1,
                  module = module2, generator = generator2,
                  inverter = inverter)
show(prod2)
\end{lstlisting}

\begin{verbatim}
Object of class  ProdGCPV 

Source of meteorological information: prom- 

Latitude of source:  40.4 degrees
Latitude for calculations:  40.4 degrees

Monthly avarages:
        Dates         Eac        Edc          Yf
       <char>       <num>      <num>       <num>
 1: Jan. 2013 0.000000000 0.00000000 0.000000000
 2: Feb. 2013 0.000000000 0.00000000 0.000000000
 3: Mar. 2013 0.000000000 0.00000000 0.000000000
 4: Apr. 2013 0.000000000 0.00000000 0.000000000
 5: May. 2013 0.000000000 0.00000000 0.000000000
 6: Jun. 2013 0.000000000 0.00000000 0.000000000
 7: Jul. 2013 0.007712197 0.02014982 0.007007584
 8: Aug. 2013 0.026013451 0.06349067 0.023636773
 9: Sep. 2013 0.000000000 0.00000000 0.000000000
10: Oct. 2013 0.000000000 0.00000000 0.000000000
11: Nov. 2013 0.000000000 0.00000000 0.000000000
12: Dec. 2013 0.000000000 0.00000000 0.000000000

Yearly values:
   Dates      Eac      Edc        Yf
   <int>    <num>    <num>     <num>
1:  2013 1.045495 2.592855 0.9499751
-----------------
Mode of tracking:  fixed 
    Inclination:  30 
    Orientation:  -19 
-----------------
Generator:
    Modules in series:  5 
    Modules in parallel:  1 
    Nominal power (kWp):  1.1
\end{verbatim}

\section{Comparación entre los tres}
\label{sec:org8c6ef73}
