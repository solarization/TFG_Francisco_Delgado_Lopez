\chapter{Código completo}

Todo el código que se muestra a continuación está disponible\ldots{}

\section{Constructores}
\label{sec:orgbf5cd38}
\subsection{calcSol}
\label{sec:org542a2bd}
\begin{lstlisting}[numbers=left,language=r,label= ,caption= ,captionpos=b]
calcSol <- function(lat, BTd,
                    sample = 'hour', BTi,
                    EoT = TRUE,
                    keep.night = TRUE,
                    method = 'michalsky')
{
    if(missing(BTd)) BTd <- truncDay(BTi)
    solD <- fSolD(lat, BTd, method = method) #daily values
    solI <- fSolI(solD = solD, sample = sample, #intradaily values
                  BTi = BTi, keep.night = keep.night,
                  EoT = EoT, method = method)

    if(!missing(BTi)){
        sample <- solI$Dates[2]-solI$Dates[1]
        sample <- format(sample)
    }

    solD[, lat := NULL]
    solI[, lat := NULL]
    result <- new('Sol',
                  lat = lat,
                  solD = solD,
                  solI = solI,
                  sample = sample,
                  method = method)
    return(result)
}
\end{lstlisting}

\subsection{calcG0}
\label{sec:orgf7ccb51}
\begin{lstlisting}[numbers=left,language=r,label= ,caption= ,captionpos=b]
calcG0 <- function(lat,
                   modeRad='prom',
                   dataRad,
                   sample='hour',
                   keep.night=TRUE,
                   sunGeometry='michalsky',
                   corr, f, ...)
{

    if (missing(lat)) stop('lat missing. You must provide a latitude value.')

    stopifnot(modeRad %in% c('prom', 'aguiar','bd', 'bdI'))


###Datos de Radiacion
    if (missing(corr)){
        corr = switch(modeRad,
                      bd = 'CPR', #Correlation between Fd and Kt for daily values
                      aguiar = 'CPR', #Correlation between Fd and Kt for daily values
                      prom = 'Page',  #Correlation between Fd and Kt for monthly averages
                      bdI = 'BRL'     #Correlation between fd and kt for intraday values
                      )
    }

    if(is(dataRad, 'Meteo')){BD <- dataRad}
    else{
    BD <- switch(modeRad,
                 bd = {
                         if (!is.list(dataRad)) dataRad <- list(file=dataRad)
                         switch(class(dataRad$file)[1],
                                character={
                                    bd.default=list(file='', lat=lat)
                                    bd=modifyList(bd.default, dataRad)
                                    res <- do.call('readBDd', bd)
                                    res
                                },
                                data.table= ,
                                data.frame={
                                    bd.default=list(file='', lat=lat)
                                    bd=modifyList(bd.default, dataRad)
                                    res <- do.call('dt2Meteo', bd)
                                    res
                                },
                                zoo={
                                    bd.default=list(file='', lat=lat, source='')
                                    bd=modifyList(bd.default, dataRad)
                                    res <- do.call('zoo2Meteo', bd)
                                    res
                                })
                     }, #End of bd
                 prom = {
                     if (!is.list(dataRad)) dataRad <- list(G0dm=dataRad)
                     prom.default <- list(G0dm=numeric(), lat=lat)
                     prom = modifyList(prom.default, dataRad)
                     res <- do.call('readG0dm', prom)
                 }, #End of prom
                 aguiar = {
                     if (is.list(dataRad)) dataRad <- dataRad$G0dm
                     BTd <- fBTd(mode='serie')
                     solD <- fSolD(lat, BTd)
                     G0d <- markovG0(dataRad, solD)
                     res <- dt2Meteo(G0d, lat=lat, source='aguiar')
                 }, #End of aguiar
                 bdI = {
                     if (!is.list(dataRad)) dataRad <- list(file=dataRad)
                     switch(class(dataRad$file)[1],
                            character = {
                                bdI.default <- list(file='', lat=lat)
                                bdI <- modifyList(bdI.default, dataRad)
                                res <- do.call('readBDi', bdI)
                                res
                            },
                            data.table = ,
                            data.frame = {
                                bdI.default <- list(file='', lat=lat)
                                bdI <- modifyList(bdI.default, dataRad)
                                res <- do.call('dt2Meteo', bdI)
                                res
                            },
                            zoo = {
                                bdI.default <- list(file='', lat=lat, source='')
                                bdI <- modifyList(bdI.default, dataRad)
                                res <- do.call('zoo2Meteo', bdI)
                                res
                            },
                            stop('dataRad$file should be a character, a data.table, a data.frame or a zoo.')
                            )} #End of btI
                 )             #End of general switch    
    }


### Angulos solares y componentes de irradiancia
    if (modeRad=='bdI') {
        sol <- calcSol(lat, sample = sample,
                       BTi = indexD(BD), keep.night=keep.night, method=sunGeometry)
        compI <- fCompI(sol=sol, G0I=BD, corr=corr, f=f, ...)
        compD <- compI[, lapply(.SD, P2E, sol@sample),
                       .SDcols = c('G0', 'D0', 'B0'),
                       by = truncDay(Dates)]
        names(compD)[1] <- 'Dates'
        names(compD)[-1] <- paste(names(compD)[-1], 'd', sep = '')
        compD$Fd <- compD$D0d/compD$G0d
        compD$Kt <- compD$G0d/sol@solD$Bo0d
    } else { ##modeRad!='bdI'
        sol <- calcSol(lat, indexD(BD), sample = sample,
                       keep.night = keep.night, method = sunGeometry)
        compD<-fCompD(sol=sol, G0d=BD, corr=corr, f, ...)
        compI<-fCompI(sol=sol, compD=compD, ...)
    }

###Temperature

    Ta=switch(modeRad,
              bd={
                  if (all(c("TempMax","TempMin") %in% names(BD@data))) {
                      fTemp(sol, BD)
                  } else {
                      if ("Ta" %in% names(BD@data)) {
                          data.table(Dates = indexD(sol),
                                     Ta =BD@data$Ta)
                      } else {
                          warning('No temperature information available!')
                      }
                  }
              },
              bdI={
                  if ("Ta" %in% names(BD@data)) {
                      data.table(Dates = indexI(sol),
                                 Ta = BD@data$Ta)
                  } else {
                      warning('No temperature information available!')
                  }
              },
              prom={
                  if ("Ta" %in% names(BD@data)) {
                      data.table(Dates = indexD(sol),
                                 Ta = BD@data$Ta)
                  } else {
                      warning('No temperature information available!')
                  }                  
              },
              aguiar={ 
                  data.table(Dates = indexI(sol),
                             Ta = BD@data$Ta)
              }
              )

###Medias mensuales y anuales
    nms <- c('G0d', 'D0d', 'B0d')
    G0dm <- compD[, lapply(.SD/1000, mean, na.rm = TRUE),
                  .SDcols = nms,
                  by = .(month(Dates), year(Dates))]

    if(modeRad == 'prom'){
        G0dm[, DayOfMonth := DOM(G0dm)]
        G0y <- G0dm[, lapply(.SD*DayOfMonth, sum, na.rm = TRUE),
                    .SDcols = nms,
                    by = .(Dates = year)]
        G0dm[, DayOfMonth := NULL]        
    } else{
        G0y <- compD[, lapply(.SD/1000, sum, na.rm = TRUE),
                     .SDcols = nms,
                     by = .(Dates = year(Dates))]
    }
    G0dm[, Dates := paste(month.abb[month], year, sep = '. ')]
    G0dm[, c('month', 'year') := NULL]
    setcolorder(G0dm, 'Dates')

###Result
    result <- new(Class='G0',
                  BD,        #G0 contains "Meteo"
                  sol,       #G0 contains 'Sol'
                  G0D=compD, #results of fCompD
                  G0dm=G0dm, #monthly means
                  G0y=G0y,   #yearly values
                  G0I=compI, #results of fCompD
                  Ta=Ta      #ambient temperature
                  )
    return(result)
}
\end{lstlisting}

\subsection{calcGef}
\label{sec:orga29da02}
\begin{lstlisting}[numbers=left,language=r,label= ,caption= ,captionpos=b]
calcGef<-function(lat,
                  modeTrk='fixed',      #c('two','horiz','fixed')
                  modeRad='prom', 
                  dataRad,
                  sample='hour',
                  keep.night=TRUE,
                  sunGeometry='michalsky',
                  corr, f,
                  betaLim=90, beta=abs(lat)-10, alfa=0,
                  iS=2, alb=0.2, horizBright=TRUE, HCPV=FALSE,
                  modeShd='',    #modeShd=c('area','bt','prom')
                  struct=list(), #list(W=23.11, L=9.8, Nrow=2, Ncol=8), 
                  distances=data.frame(),#data.table(Lew=40, Lns=30, H=0)){
                  ...){

    stopifnot(is.list(struct), is.data.frame(distances))

    if (('bt' %in% modeShd) & (modeTrk!='horiz')) {
        modeShd[which(modeShd=='bt')]='area'
        warning('backtracking is only implemented for modeTrk=horiz')}

    if (modeRad!='prev'){ #not use a prev calculation
        radHoriz <- calcG0(lat=lat, modeRad=modeRad,
                           dataRad=dataRad,
                           sample=sample, keep.night=keep.night,
                           sunGeometry=sunGeometry,
                           corr=corr, f=f, ...)
    } else {                          #use a prev calculation
        radHoriz <- as(dataRad, 'G0') 
    } 

### Inclined and effective radiation
    BT=("bt" %in% modeShd) 
    angGen <- fTheta(radHoriz, beta, alfa, modeTrk, betaLim, BT, struct, distances)
    inclin <- fInclin(radHoriz, angGen, iS, alb, horizBright, HCPV)

### Daily, monthly and yearly values
    by <- radHoriz@sample
    nms <- c('Bo', 'Bn', 'G', 'D', 'B', 'Gef', 'Def', 'Bef')
    nmsd <- paste(nms, 'd', sep = '')


    if(radHoriz@type == 'prom'){
        Gefdm <- inclin[, lapply(.SD/1000, P2E, by),
                        .SDcols = nms,
                        by = .(month(Dates), year(Dates))]
        names(Gefdm)[-c(1,2)] <- nmsd
        GefD <- Gefdm[, .SD*1000,
                      .SDcols = nmsd,
                      by = .(Dates = indexD(radHoriz))]

        Gefdm[, DayOfMonth := DOM(Gefdm)]
        Gefy <- Gefdm[, lapply(.SD*DayOfMonth, sum, na.rm = TRUE),
                      .SDcols = nmsd,
                      by = .(Dates = year)]
        Gefdm[, DayOfMonth := NULL]
    } else{
        GefD <- inclin[, lapply(.SD, P2E, by),
                       .SDcols = nms,
                       by = .(Dates = truncDay(Dates))]
        names(GefD)[-1] <- nmsd

        Gefdm <- GefD[, lapply(.SD/1000, mean, na.rm = TRUE),
                      .SDcols = nmsd,
                      by = .(month(indexD(radHoriz)), year(indexD(radHoriz)))]
        Gefy <- GefD[, lapply(.SD/1000, sum, na.rm = TRUE),
                     .SDcols = nmsd,
                     by = .(Dates = year(indexD(radHoriz)))]
    }

    Gefdm[, Dates := paste(month.abb[month], year, sep = '. ')]
    Gefdm[, c('month', 'year') := NULL]
    setcolorder(Gefdm, 'Dates')

###Resultado antes de sombras
    result0=new('Gef',
                radHoriz,                           #Gef contains 'G0'
                Theta=angGen,
                GefD=GefD,
                Gefdm=Gefdm,
                Gefy=Gefy,
                GefI=inclin,
                iS=iS,
                alb=alb,
                modeTrk=modeTrk,
                modeShd=modeShd,
                angGen=list(alfa=alfa, beta=beta, betaLim=betaLim),
                struct=struct,
                distances=distances
                )
###Shadows
    if (isTRUE(modeShd == "") ||        #If modeShd=='' there is no shadow calculation
        ('bt' %in% modeShd)) {            #nor if there is backtracking
        return(result0)
    } else {
        result <- calcShd(result0, modeTrk, modeShd, struct, distances)
        return(result)
    }
}
\end{lstlisting}

\subsection{prodGCPV}
\label{sec:org2efb3d9}
\begin{lstlisting}[numbers=left,language=r,label= ,caption= ,captionpos=b]
prodGCPV<-function(lat,
                   modeTrk='fixed', 
                   modeRad='prom',
                   dataRad,
                   sample='hour',
                   keep.night=TRUE,
                   sunGeometry='michalsky',
                   corr, f,
                   betaLim=90, beta=abs(lat)-10, alfa=0,
                   iS=2, alb=0.2, horizBright=TRUE, HCPV=FALSE,
                   module=list(), 
                   generator=list(),
                   inverter=list(), 
                   effSys=list(), 
                   modeShd='',    
                   struct=list(), 
                   distances=data.table(),
                   ...){

    stopifnot(is.list(module),
              is.list(generator),
              is.list(inverter),
              is.list(effSys),
              is.list(struct),
              is.data.table(distances))

  if (('bt' %in% modeShd) & (modeTrk!='horiz')) {
      modeShd[which(modeShd=='bt')]='area'
      warning('backtracking is only implemented for modeTrk=horiz')}

    if (modeRad!='prev'){ #We do not use a previous calculation

    radEf<-calcGef(lat=lat, modeTrk=modeTrk, modeRad=modeRad,
                   dataRad=dataRad,
                   sample=sample, keep.night=keep.night,
                   sunGeometry=sunGeometry,
                   corr=corr, f=f,
                   betaLim=betaLim, beta=beta, alfa=alfa,
                   iS=iS, alb=alb, horizBright=horizBright, HCPV=HCPV,
                   modeShd=modeShd, struct=struct, distances=distances, ...)

  } else { #We use a previous calcG0, calcGef or prodGCPV calculation.

      stopifnot(class(dataRad) %in% c('G0', 'Gef', 'ProdGCPV'))
      radEf <- switch(class(dataRad),
                      G0=calcGef(lat=lat,
                                 modeTrk=modeTrk, modeRad='prev',
                                 dataRad=dataRad,
                                 betaLim=betaLim, beta=beta, alfa=alfa,
                                 iS=iS, alb=alb, horizBright=horizBright, HCPV=HCPV,
                                 modeShd=modeShd, struct=struct, distances=distances, ...),
                      Gef=dataRad,
                      ProdGCPV=as(dataRad, 'Gef')
                      )
  }


    ##Production
    prodI<-fProd(radEf,module,generator,inverter,effSys)
    module=attr(prodI, 'module')
    generator=attr(prodI, 'generator')
    inverter=attr(prodI, 'inverter')
    effSys=attr(prodI, 'effSys')

    ##Calculation of daily, monthly and annual values
    Pg=generator$Pg #Wp

    by <- radEf@sample
    nms1 <- c('Pac', 'Pdc')
    nms2 <- c('Eac', 'Edc', 'Yf')


    if(radEf@type == 'prom'){
        prodDm <- prodI[, lapply(.SD/1000, P2E, by),
                        .SDcols = nms1,
                        by = .(month(Dates), year(Dates))]
        names(prodDm)[-c(1,2)] <- nms2[-3]
        prodDm[, Yf := Eac/(Pg/1000)]
        prodD <- prodDm[, .SD*1000,
                        .SDcols = nms2,
                        by = .(Dates = indexD(radEf))]
        prodD[, Yf := Yf/1000]

        prodDm[, DayOfMonth := DOM(prodDm)]
        prody <- prodDm[, lapply(.SD*DayOfMonth, sum, na.rm = TRUE),
                        .SDcols = nms2,
                        by = .(Dates = year)]
        prodDm[, DayOfMonth := NULL]
    } else {
        prodD <- prodI[, lapply(.SD, P2E, by),
                       .SDcols = nms1,
                       by = .(Dates = truncDay(Dates))]
        names(prodD)[-1] <- nms2[-3]
        prodD[, Yf := Eac/Pg]

        prodDm <- prodD[, lapply(.SD/1000, mean, na.rm = TRUE),
                        .SDcols = nms2,
                        by = .(month(Dates), year(Dates))]
        prodDm[, Yf := Yf * 1000]
        prody <- prodD[, lapply(.SD/1000, sum, na.rm = TRUE),
                       .SDcols = nms2,
                       by = .(Dates = year(Dates))]
        prody[, Yf := Yf * 1000]
    }

    prodDm[, Dates := paste(month.abb[month], year, sep = '. ')]
    prodDm[, c('month', 'year') := NULL]
    setcolorder(prodDm, 'Dates')

    result <- new('ProdGCPV',
                  radEf,                  #contains 'Gef'
                  prodD=prodD,
                  prodDm=prodDm,
                  prody=prody,
                  prodI=prodI,
                  module=module,
                  generator=generator,
                  inverter=inverter,
                  effSys=effSys
                  )
}
\end{lstlisting}

\subsection{prodPVPS}
\label{sec:orgfea25b9}
\begin{lstlisting}[numbers=left,language=r,label= ,caption= ,captionpos=b]
prodPVPS<-function(lat, 
                   modeTrk='fixed', 
                   modeRad='prom', 
                   dataRad,
                   sample='hour',
                   keep.night=TRUE,
                   sunGeometry='michalsky',
                   corr, f,
                   betaLim=90, beta=abs(lat)-10, alfa=0,
                   iS=2, alb=0.2, horizBright=TRUE, HCPV=FALSE,
                   pump , H, 
                   Pg, converter= list(), #Pnom=Pg, Ki=c(0.01,0.025,0.05)),
                   effSys=list(),
                   ...){

    stopifnot(is.list(converter),
              is.list(effSys))

    if (modeRad!='prev'){ #We do not use a previous calculation

        radEf<-calcGef(lat=lat, modeTrk=modeTrk, modeRad=modeRad,
                       dataRad=dataRad,
                       sample=sample, keep.night=keep.night,
                       sunGeometry=sunGeometry,
                       corr=corr, f=f,
                       betaLim=betaLim, beta=beta, alfa=alfa,
                       iS=iS, alb=alb, horizBright=horizBright, HCPV=HCPV,
                   modeShd='', ...)

    } else { #We use a previous calculation of calcG0, calcGef or prodPVPS
        stopifnot(class(dataRad) %in% c('G0', 'Gef', 'ProdPVPS'))
        radEf <- switch(class(dataRad),
                        G0=calcGef(lat=lat, 
                                   modeTrk=modeTrk, modeRad='prev',
                      dataRad=dataRad,
                      betaLim=betaLim, beta=beta, alfa=alfa,
                      iS=iS, alb=alb, horizBright=horizBright, HCPV=HCPV,
                      modeShd='', ...),
                      Gef=dataRad,
                      ProdPVPS=as(dataRad, 'Gef')
                      )
    }

###Electric production
    converter.default=list(Ki = c(0.01,0.025,0.05), Pnom=Pg)
    converter=modifyList(converter.default, converter)

    effSys.default=list(ModQual=3,ModDisp=2,OhmDC=1.5,OhmAC=1.5,MPP=1,TrafoMT=1,Disp=0.5)
    effSys=modifyList(effSys.default, effSys)

    TONC=47
    Ct=(TONC-20)/800
    lambda=0.0045
    Gef=radEf@GefI$Gef
    night=radEf@solI$night
    Ta=radEf@Ta$Ta

    Tc=Ta+Ct*Gef
    Pdc=Pg*Gef/1000*(1-lambda*(Tc-25))
    Pdc[is.na(Pdc)]=0 #Necessary for the functions provided by fPump
    PdcN=with(effSys,
              Pdc/converter$Pnom*(1-ModQual/100)*(1-ModDisp/100)*(1-OhmDC/100)
              )
    PacN=with(converter,{
        A=Ki[3]
        B=Ki[2]+1
        C=Ki[1]-(PdcN)
        ##AC power normalized to the inverter
        result=(-B+sqrt(B^2-4*A*C))/(2*A)
    })
    PacN[PacN<0]<-0

    Pac=with(converter,
             PacN*Pnom*(1-effSys$OhmAC/100))
    Pdc=PdcN*converter$Pnom*(Pac>0)


###Pump
    fun<-fPump(pump=pump, H=H)
    ##I limit power to the pump operating range.
    rango=with(fun,Pac>=lim[1] & Pac<=lim[2]) 
    Pac[!rango]<-0
    Pdc[!rango]<-0
    prodI=data.table(Pac=Pac,Pdc=Pdc,Q=0,Pb=0,Ph=0,f=0)	
    prodI=within(prodI,{
        Q[rango]<-fun$fQ(Pac[rango])
        Pb[rango]<-fun$fPb(Pac[rango])
        Ph[rango]<-fun$fPh(Pac[rango])
        f[rango]<-fun$fFreq(Pac[rango])
        etam=Pb/Pac
        etab=Ph/Pb
    })

    prodI[night,]<-NA
    prodI[, Dates := indexI(radEf)]
    setcolorder(prodI, c('Dates', names(prodI)[-length(prodI)]))

###daily, monthly and yearly values

    by <- radEf@sample

    if(radEf@type == 'prom'){
        prodDm <- prodI[, .(Eac = P2E(Pac, by)/1000,
                            Qd = P2E(Q, by)),
                        by = .(month(Dates), year(Dates))]
        prodDm[, Yf := Eac/(Pg/1000)]

        prodD <- prodDm[, .(Eac = Eac*1000,
                            Qd,
                            Yf),
                        by = .(Dates = indexD(radEf))]

        prodDm[, DayOfMonth := DOM(prodDm)]

        prody <- prodDm[, lapply(.SD*DayOfMonth, sum, na.rm = TRUE),
                        .SDcols = c('Eac', 'Qd', 'Yf'),
                        by = .(Dates = year)]
        prodDm[, DayOfMonth := NULL]
    } else {
        prodD <- prodI[, .(Eac = P2E(Pac, by)/1000,
                           Qd = P2E(Q, by)),
                       by = .(Dates = truncDay(Dates))]
        prodD[, Yf := Eac/Pg*1000]

        prodDm <- prodD[, lapply(.SD, mean, na.rm = TRUE),
                        .SDcols = c('Eac','Qd', 'Yf'),
                        by = .(month(Dates), year(Dates))]
        prody <- prodD[, lapply(.SD, sum, na.rm = TRUE),
                       .SDcols = c('Eac', 'Qd', 'Yf'),
                       by = .(Dates = year(Dates))]

    }

    prodDm[, Dates := paste(month.abb[month], year, sep = '. ')]
    prodDm[, c('month', 'year') := NULL]
    setcolorder(prodDm, 'Dates')

    result <- new('ProdPVPS',
                  radEf,                  #contains 'Gef'
                  prodD=prodD,
                  prodDm=prodDm,
                  prody=prody,
                  prodI=prodI,
                  pump=pump,
                  H=H,
                  Pg=Pg,
                  converter=converter,
                  effSys=effSys
                )
}
\end{lstlisting}

\subsection{calcShd}
\label{sec:org18bc738}
\begin{lstlisting}[numbers=left,language=r,label= ,caption= ,captionpos=b]
calcShd<-function(radEf,##class='Gef'
                  modeTrk='fixed',     #c('two','horiz','fixed')
                  modeShd='prom',      #modeShd=c('area','bt','prom')
                  struct=list(), #list(W=23.11, L=9.8, Nrow=2, Ncol=8), 
                  distances=data.frame() #data.table(Lew=40, Lns=30, H=0)){
                  )
{
    stopifnot(is.list(struct), is.data.frame(distances))

    ##For now I only use modeShd = 'area'
    ##With different modeShd (to be defined) I will be able to calculate Gef in a different way
    ##See macagnan thesis
    prom=("prom"  %in%  modeShd)
    prev <- as.data.tableI(radEf, complete=TRUE)
    ## shadow calculations
    sol <- data.table(AzS = prev$AzS,
                      AlS = prev$AlS)
    theta <- radEf@Theta
    AngGen <- data.table(theta, sol)
    FS <- fSombra(AngGen, distances, struct, modeTrk, prom)
    ## irradiance calculation
    gef0 <- radEf@GefI
    Bef0 <- gef0$Bef
    Dcef0 <- gef0$Dcef
    Gef0 <- gef0$Gef
    Dief0 <- gef0$Dief
    Ref0 <- gef0$Ref
    ## calculation
    Bef <- Bef0*(1-FS)
    Dcef <- Dcef0*(1-FS)
    Def <- Dief0+Dcef
    Gef <- Dief0+Ref0+Bef+Dcef #Including shadows
    ##Change names
    nms <- c('Gef', 'Def', 'Dcef', 'Bef')
    nmsIndex <- which(names(gef0) %in% nms)
    names(gef0)[nmsIndex]<- paste(names(gef0)[nmsIndex], '0', sep='')
    GefShd <- gef0
    GefShd[, c(nms, 'FS') := .(Gef, Def, Dcef, Bef, FS)]

    ## daily, monthly and yearly values
    by <- radEf@sample
    nms <- c('Gef0', 'Def0', 'Bef0', 'G', 'D', 'B', 'Gef', 'Def', 'Bef')
    nmsd <- paste(nms, 'd', sep = '')

    Gefdm <- GefShd[, lapply(.SD/1000, P2E, by),
                    by = .(month(truncDay(Dates)), year(truncDay(Dates))),
                    .SDcols = nms]
    names(Gefdm)[-c(1, 2)] <- nmsd

    if(radEf@type == 'prom'){
        GefD <- Gefdm[, .SD[, -c(1, 2)] * 1000,
                      .SDcols = nmsd,
                      by = .(Dates = indexD(radEf))] 

        Gefdm[, DayOfMonth := DOM(Gefdm)]

        Gefy <- Gefdm[, lapply(.SD*DayOfMonth, sum, na.rm = TRUE),
                      .SDcols = nmsd,
                      by = .(Dates = year)]
        Gefdm[, DayOfMonth := NULL]
    } else{    
        GefD <- GefShd[, lapply(.SD/1000, P2E, by),
                       .SDcols = nms,
                       by = .(Dates = truncDay(Dates))]
        names(GefD)[-1] <- nmsd

        Gefy <- GefD[, lapply(.SD[, -1], sum, na.rm = TRUE),
                     .SDcols = nmsd,
                     by = .(Dates = year(Dates))]
    }

    Gefdm[, Dates := paste(month.abb[month], year, sep = '. ')]
    Gefdm[, c('month', 'year') := NULL]
    setcolorder(Gefdm, c('Dates', names(Gefdm)[-length(Gefdm)]))

    ## Object of class Gef
    ## modifying the 'modeShd', 'GefI', 'GefD', 'Gefdm', and 'Gefy' slots
    ## from the original radEf object
    radEf@modeShd=modeShd
    radEf@GefI=GefShd
    radEf@GefD=GefD
    radEf@Gefdm=Gefdm
    radEf@Gefy=Gefy
    return(radEf)
}
\end{lstlisting}

\subsection{optimShd}
\label{sec:orgf876829}
\begin{lstlisting}[numbers=left,language=r,label= ,caption= ,captionpos=b]
optimShd<-function(lat,
                   modeTrk='fixed', 
                   modeRad='prom', 
                   dataRad,
                   sample='hour',
                   keep.night=TRUE,
                   sunGeometry='michalsky',
                   betaLim=90, beta=abs(lat)-10, alfa=0,
                   iS=2, alb=0.2, HCPV=FALSE,
                   module=list(), 
                   generator=list(),
                   inverter=list(), 
                   effSys=list(), 
                   modeShd='',    
                   struct=list(), 
                   distances=data.table(),
                   res=2,      #resolution, distance spacing
                   prog=TRUE){ #Drawing progress bar

    if (('bt' %in% modeShd) & (modeTrk!='horiz')) {
        modeShd[which(modeShd=='bt')]='area'
        warning('backtracking is only implemented for modeTrk=horiz')}

    ##I save function arguments for later use

    listArgs<-list(lat=lat, modeTrk=modeTrk, modeRad=modeRad,
                   dataRad=dataRad,
                   sample=sample, keep.night=keep.night,
                   sunGeometry=sunGeometry,
                   betaLim=betaLim, beta=beta, alfa=alfa,
                   iS=iS, alb=alb, HCPV=HCPV,
                   module=module, generator=generator,
                   inverter=inverter, effSys=effSys,
                   modeShd=modeShd, struct=struct,
                   distances=data.table(Lew=NA, Lns=NA, D=NA))


    ##I think network on which I will do the calculations
    Red=switch(modeTrk,
               horiz=with(distances,
                          data.table(Lew=seq(Lew[1],Lew[2],by=res),
                                     H=0)),
               two=with(distances,
                        data.table(
                        expand.grid(Lew=seq(Lew[1],Lew[2],by=res),
                                    Lns=seq(Lns[1],Lns[2],by=res),
                                    H=0))),
               fixed=with(distances,
                          data.table(D=seq(D[1],D[2],by=res),
                                     H=0))
    )

    casos<-dim(Red)[1] #Number of possibilities to study

    ##I prepare the progress bar
    if (prog) {pb <- txtProgressBar(min = 0, max = casos+1, style = 3)
        setTxtProgressBar(pb, 0)}

###Calculations	
    ##Reference: No shadows	
    listArgs0 <- modifyList(listArgs,
                            list(modeShd='', struct=NULL, distances=NULL) )
    Prod0<-do.call(prodGCPV, listArgs0)
    YfAnual0=mean(Prod0@prody$Yf) #I use mean in case there are several years
    if (prog) {setTxtProgressBar(pb, 1)}

    ##The loop begins

    ##I create an empty vector of the same length as the cases to be studied
    YfAnual<-numeric(casos) 

    BT=('bt' %in% modeShd)
    if (BT) { ##There is backtracking, then I must start from horizontal radiation.
        RadBT <- as(Prod0, 'G0')
        for (i in seq_len(casos)){
            listArgsBT <- modifyList(listArgs,
                                     list(modeRad='prev', dataRad=RadBT,
                                          distances=Red[i,]))
            prod.i <- do.call(prodGCPV, listArgsBT)
            YfAnual[i]=mean(prod.i@prody$Yf)
            if (prog) {setTxtProgressBar(pb, i+1)}
        }
    } else {
        prom=('prom' %in% modeShd)
        for (i in seq_len(casos)){
            Gef0=as(Prod0, 'Gef')
            GefShd=calcShd(Gef0, modeTrk=modeTrk, modeShd=modeShd,
                           struct=struct, distances=Red[i,])
            listArgsShd <- modifyList(listArgs,
                                      list(modeRad='prev', dataRad=GefShd)
                                      )
            prod.i <- do.call(prodGCPV, listArgsShd)
            YfAnual[i]=mean(prod.i@prody$Yf)
            if (prog) {setTxtProgressBar(pb, i+1)}
        }
    }
    if (prog) {close(pb)}


###Results
    FS=1-YfAnual/YfAnual0
    GRR=switch(modeTrk,
               two=with(Red,Lew*Lns)/with(struct,L*W),
               fixed=Red$D/struct$L,
               horiz=Red$Lew/struct$L)
    SombraDF=data.table(Red,GRR,FS,Yf=YfAnual)
    FS.loess=switch(modeTrk,
                    two=loess(FS~Lew*Lns,data=SombraDF),
                    horiz=loess(FS~Lew,data=SombraDF),
                    fixed=loess(FS~D,data=SombraDF))
    Yf.loess=switch(modeTrk,
                    two=loess(Yf~Lew*Lns,data=SombraDF),
                    horiz=loess(Yf~Lew,data=SombraDF),
                    fixed=loess(Yf~D,data=SombraDF))
    result <- new('Shade',
                  Prod0, ##contains ProdGCPV
                  FS=FS,
                  GRR=GRR,
                  Yf=YfAnual,
                  FS.loess=FS.loess,
                  Yf.loess=Yf.loess,
                  modeShd=modeShd,
                  struct=struct,
                  distances=Red,
                  res=res
                  )
    result
}
\end{lstlisting}

\subsection{meteoReaders}
\label{sec:orgbb6f5fe}
\begin{lstlisting}[numbers=left,language=r,label= ,caption= ,captionpos=b]
#### monthly means of irradiation ####
readG0dm <- function(G0dm, Ta = 25, lat = 0,
                     year = as.POSIXlt(Sys.Date())$year + 1900,
                     promDays = c(17, 14, 15, 15, 15, 10, 18, 18, 18, 19, 18, 13),
                     source = '')
{
    if(missing(lat)){lat <- 0}
    Dates <- as.IDate(paste(year, 1:12, promDays, sep = '-'), tz = 'UTC')
    G0dm.dt <- data.table(Dates = Dates,
                          G0d = G0dm,
                          Ta = Ta)
    setkey(G0dm.dt, 'Dates')
    results <- new(Class = 'Meteo',
                   latm = lat,
                   data = G0dm.dt,
                   type = 'prom',
                   source = source)
}

#### file to Meteo (daily) ####
readBDd <- function(file, lat,
                   format = "%d/%m/%Y",header = TRUE,
                   fill = TRUE, dec = '.', sep = ';',
                   dates.col = 'Dates', ta.col = 'Ta',
                   g0.col = 'G0', keep.cols = FALSE)
{
    #stops if the arguments are not characters or numerics
    stopifnot(is.character(dates.col) || is.numeric(dates.col))
    stopifnot(is.character(ta.col) || is.numeric(ta.col))
    stopifnot(is.character(g0.col) || is.numeric(g0.col))

    #read from file and set it in a data.table
    bd <- fread(file, header = header, fill = fill, dec = dec, sep = sep)

    #check the columns
    if(!(dates.col %in% names(bd))) stop(paste('The column', dates.col, 'is not in the file'))
    if(!(g0.col %in% names(bd))) stop(paste('The column', g0.col, 'is not in the file'))
    if(!(ta.col %in% names(bd))) stop(paste('The column', ta.col, 'is not in the file'))

    #name the dates column by Dates
    Dates <- bd[[dates.col]]
    bd[,(dates.col) := NULL]
    bd[, Dates := as.IDate(Dates, format = format)]

    #name the g0 column by G0
    G0 <- bd[[g0.col]]
    bd[, (g0.col) := NULL]
    bd[, G0 := as.numeric(G0)]

    #name the ta column by Ta
    Ta <- bd[[ta.col]]
    bd[, (ta.col) := NULL]
    bd[, Ta := as.numeric(Ta)]

    names0 <- NULL
    if(all(c('D0', 'B0') %in% names(bd))){
        names0 <- c(names0, 'D0', 'B0')
    }

    names0 <- c(names0, 'Ta')

    if(all(c('TempMin', 'TempMax') %in% names(bd))){
        names0 <- c(names0, 'TempMin', 'TempMax')
    }
    if(keep.cols)
    {
        #keep the rest of the columns but reorder the columns
        setcolorder(bd, c('Dates', 'G0', names0))
    }
    else
    {
        #erase the rest of the columns
        cols <- c('Dates', 'G0', names0)
        bd <- bd[, ..cols]
    }

    setkey(bd, 'Dates')
    result <- new(Class = 'Meteo',
                  latm = lat,
                  data = bd,
                  type = 'bd',
                  source = file)
}

#### file to Meteo (intradaily) ####
readBDi <- function(file, lat,
                    format = "%d/%m/%Y %H:%M:%S",
                    header = TRUE, fill = TRUE, dec = '.',
                    sep = ';', dates.col = 'dates', times.col,
                    ta.col = 'Ta', g0.col = 'G0', keep.cols = FALSE)
{
    #stops if the arguments are not characters or numerics
    stopifnot(is.character(dates.col) || is.numeric(dates.col))
    stopifnot(is.character(ta.col) || is.numeric(ta.col))
    stopifnot(is.character(g0.col) || is.numeric(g0.col))

    #read from file and set it in a data.table
    bd <- fread(file, header = header, fill = fill, dec = dec, sep = sep)

    #check the columns
    if(!(dates.col %in% names(bd))) stop(paste('The column', dates.col, 'is not in the file'))
    if(!(g0.col %in% names(bd))) stop(paste('The column', g0.col, 'is not in the file'))
    if(!(ta.col %in% names(bd))) stop(paste('The column', ta.col, 'is not in the file'))

    if(!missing(times.col)){
        stopifnot(is.character(times.col) || is.numeric(times.col))
        if(!(times.col %in% names(bd))) stop(paste('The column', times.col, 'is not in the file'))

        #name the dates column by Dates
        format <- strsplit(format, ' ')
        dd <- as.IDate(bd[[dates.col]], format = format[[1]][1])
        tt <- as.ITime(bd[[times.col]], format = format[[1]][2])
        bd[,(dates.col) := NULL]
        bd[,(times.col) := NULL]
        bd[, Dates := as.POSIXct(dd, tt, tz = 'UTC')]
    }

    else
    {
        dd <- as.POSIXct(bd[[dates.col]], format = format, tz = 'UTC')
        bd[, (dates.col) := NULL]
        bd[, Dates := dd]
    }

    #name the g0 column by G0
    G0 <- bd[[g0.col]]
    bd[, (g0.col) := NULL]
    bd[, G0 := as.numeric(G0)]

    #name the ta column by Ta
    Ta <- bd[[ta.col]]
    bd[, (ta.col) := NULL]
    bd[, Ta := as.numeric(Ta)]

    names0 <- NULL
    if(all(c('D0', 'B0') %in% names(bd))){
        names0 <- c(names0, 'D0', 'B0')
    }

    names0 <- c(names0, 'Ta')

    if(keep.cols)
    {
        #keep the rest of the columns but reorder the columns
        setcolorder(bd, c('Dates', 'G0', names0))
    }
    else
    {
        #erase the rest of the columns
        cols <- c('Dates', 'G0', names0)
        bd <- bd[, ..cols]
    }

    setkey(bd, 'Dates')
    result <- new(Class = 'Meteo',
                  latm = lat,
                  data = bd,
                  type = 'bdI',
                  source = file)
}


dt2Meteo <- function(file, lat, source = '', type){
    ## Make sure its a data.table
    bd <- data.table(file)

    ## Dates is an as.POSIX element
    bd[, Dates := as.POSIXct(Dates, tz = 'UTC')]

    ## type
    if(missing(type)){
        sample <- median(diff(file$Dates))
        IsDaily <- as.numeric(sample, units = 'days')
        if(is.na(IsDaily)) IsDaily <- ifelse('G0d' %in% names(bd),
                                             1, 0)
        if(IsDaily >= 30) type <- 'prom'
        else{
            type <- ifelse(IsDaily >= 1, 'bd', 'bdI') 
        }

    }
    if(!('Ta' %in% names(bd))){
        if(all(c('Tempmin', 'TempMax') %in% names(bd)))
            bd[, Ta := mean(c(Tempmin, TempMax))]
        else bd[, Ta := 25]
            }

    ## Columns of the data.table
    nms0 <- switch(type,
                   bd = ,
                   prom = {
                       nms0 <- 'G0d'
                       if(all(c('D0d', 'B0d') %in% names(bd))){
                           nms0 <- c(nms0, 'D0d', 'B0d')
                       }
                       nms0 <- c(nms0, 'Ta')
                       if(all(c('TempMin', 'TempMax') %in% names(bd))){
                           nms0 <- c(nms0, 'TempMin', 'TempMax')
                       }
                       nms0
                   },
                   bdI = {
                       nms0 <- 'G0'
                       if(all(c('D0', 'B0') %in% names(bd))){
                           nms0 <- c(nms0, 'D0', 'B0')
                       }
                       if('Ta' %in% names(bd)){
                           nms0 <- c(nms0, 'Ta')
                       }
                       nms0
                   })
    ## Columns order and set key
    setcolorder(bd, c('Dates', nms0))
    setkey(bd, 'Dates')
    ## Result
    result <- new(Class = 'Meteo',
                  latm = lat,
                  data = bd,
                  type = type,
                  source = source)
}

#### Liu and Jordan, Collares-Pereira and Rabl proposals ####
collper <- function(sol, compD)
{
    ind.rep <- indexRep(sol)
    solI <- as.data.tableI(sol, complete = T)
    ws <- solI$ws
    w <- solI$w

    a <- 0.409-0.5016*sin(ws+pi/3)
    b <- 0.6609+0.4767*sin(ws+pi/3)

    rd <- solI[, Bo0/Bo0d]
    rg <- rd * (a + b * cos(w))

    # Daily irradiation components
    G0d <- compD$G0d[ind.rep]
    B0d <- compD$B0d[ind.rep]
    D0d <- compD$D0d[ind.rep]

    # Daily profile
    G0 <- G0d * rg
    D0 <- D0d * rd

    # This method may produce diffuse irradiance higher than
    # global irradiance
    G0 <- pmax(G0, D0, na.rm = TRUE)
    B0 <- G0 - D0

    # Negative values are set to NA
    neg <- (B0 < 0) | (D0 < 0) | (G0 < 0)
    is.na(G0) <- neg
    is.na(B0) <- neg
    is.na(D0) <- neg

    # Daily profiles are scaled to keep daily irradiation values
    day <- truncDay(indexI(sol))
    sample <- sol@sample

    G0dCP <- ave(G0, day, FUN=function(x) P2E(x, sample))
    B0dCP <- ave(B0, day, FUN=function(x) P2E(x, sample))
    D0dCP <- ave(D0, day, FUN=function(x) P2E(x, sample))

    G0 <- G0 * G0d/G0dCP
    B0 <- B0 * B0d/B0dCP
    D0 <- D0 * D0d/D0dCP

    res <- data.table(G0, B0, D0)
    return(res)
}


#### intradaily Meteo to daily Meteo ####
Meteoi2Meteod <- function(G0i)
{
    lat <- G0i@latm
    source <- G0i@source

    dt0 <- getData(G0i)
    dt <- dt0[, lapply(.SD, sum), 
             .SDcols = names(dt0)[!names(dt0) %in% c('Dates', 'Ta')],
             by = .(Dates = as.IDate(Dates))]
    if('Ta' %in% names(dt0)){
        Ta <- dt0[, .(Ta = mean(Ta),
                      TempMin = min(Ta),
                      TempMax = max(Ta)),
                  by = .(Dates = as.IDate(Dates))]
        if(all(Ta$Ta == c(Ta$TempMin, Ta$TempMax))) Ta[, c('TempMin', 'TempMax') := NULL]
        dt <- merge(dt, Ta)
    }
    if('G0' %in% names(dt)){
        names(dt)[names(dt) == 'G0'] <- 'G0d'
    }
    if('D0' %in% names(dt)){
        names(dt)[names(dt) == 'D0'] <- 'D0d'
    }
    if('B0' %in% names(dt)){
        names(dt)[names(dt) == 'B0'] <- 'B0d'
    }
    G0d <- dt2Meteo(dt, lat, source, type = 'bd')
    return(G0d)
}

#### daily Meteo to monthly Meteo ####
Meteod2Meteom <- function(G0d)
{
    lat <- G0d@latm
    source <- G0d@source

    dt <- getData(G0d)
    nms <- names(dt)[-1]
    dt <- dt[, lapply(.SD, mean),
             .SDcols = nms,
             by = .(month(Dates), year(Dates))]
    dt[, Dates := fBTd()]
    dt <- dt[, c('month', 'year') := NULL]

    setcolorder(dt, 'Dates')

    G0m <- dt2Meteo(dt, lat, source, type = 'prom')
    return(G0m)
}

zoo2Meteo <- function(file, lat, source = '')
{
    sample <- median(diff(index(file)))
    IsDaily <- as.numeric(sample, units = 'days')>=1
    type <- ifelse(IsDaily, 'bd', 'bdI')
    result <- new(Class = 'Meteo',
                  latm = lat,
                  data = file,
                  type = type,
                  source = source)
}

siarGET <- function(id, inicio, final, tipo = 'Mensuales', ambito = 'Estacion'){
    if(!(tipo %in% c('Horarios', 'Diarios', 'Semanales', 'Mensuales'))){
        stop('argument \'tipo\' must be: Horarios, Diarios, Semanales or Mensuales')
    }
    if(!(ambito %in% c('CCAA', 'Provincia', 'Estacion'))){
        stop('argument \'ambito\' must be: CCAA, Provincia or Estacion')
    }

    mainURL <- "https://servicio.mapama.gob.es"

    path <- paste('/apisiar/API/v1/Datos', tipo, ambito, sep = '/')

    ## prepare the APIsiar
    req <- request(mainURL) |>
        req_url_path(path) |>
        req_url_query(Id = id,
                      FechaInicial = inicio,
                      FechaFinal = final,
                      ClaveAPI = '_Q8L_niYFBBmBs-vB3UomUqdUYy98FTRX1aYbrZ8n2FXuHYGTV')
    ## execute it
    resp <- req_perform(req)

    ##JSON to R
    respJSON <- resp_body_json(resp, simplifyVector = TRUE)

    if(!is.null(respJSON$MensajeRespuesta)){
        stop(respJSON$MensajeRespuesta)
    }

    res0 <- data.table(respJSON$Datos)

    res <- switch(tipo,
                  Horarios = {
                      res0[, HoraMin := as.ITime(sprintf('%04d', HoraMin),
                                                 format = '%H%M')]
                      res0[, Fecha := as.IDate(Fecha, format = '%Y-%m-%d')]
                      res0[, Fecha := as.IDate(ifelse(HoraMin == as.ITime(0),
                                                      Fecha+1, Fecha))]
                      res0[, Dates := as.POSIXct(HoraMin, Fecha,
                                                 tz = 'Europe/Madrid')]
                      res0 <- res0[, .(Dates,
                                       G0 = Radiacion,
                                       Ta = TempMedia)]
                      return(res0)
                  },
                  Diarios = {
                      res0[, Dates := as.IDate(Fecha)]
                      res0 <- res0[, .(Dates,
                                       G0d = Radiacion * 277.78,
                                       Ta = TempMedia,
                                       TempMin,
                                       TempMax)]
                      return(res0)
                  },
                  Semanales = res0,
                  Mensuales = {
                      promDays<-c(17,14,15,15,15,10,18,18,18,19,18,13)
                      names(res0)[1] <- 'Year'
                      res0[, Dates := as.IDate(paste(Year, Mes,
                                                     promDays[Mes],
                                                     sep = '-'))]
                      res0 <- res0[, .(Dates,
                                       G0d = Radiacion * 277.78,
                                       Ta = TempMedia,
                                       TempMin,
                                       TempMax)]
                  })

    return(res)
}

haversine <- function(lat1, lon1, lat2, lon2) {
    R <- 6371  # Radius of the Earth in kilometers
    dLat <- (lat2 - lat1) * pi / 180
    dLon <- (lon2 - lon1) * pi / 180
    a <- sin(dLat / 2) * sin(dLat / 2) + cos(lat1 * pi / 180) *
        cos(lat2 * pi / 180) * sin(dLon / 2) * sin(dLon / 2)
    c <- 2 * atan2(sqrt(a), sqrt(1 - a))
    d <- R * c
    return(d)
}

readSIAR <- function(Lon = 0, Lat = 0,
                     inicio = paste(year(Sys.Date())-1, '01-01', sep = '-'),
                     final = paste(year(Sys.Date())-1, '12-31', sep = '-'),
                     tipo = 'Mensuales', n_est = 3){
    inicio <- as.Date(inicio)
    final <- as.Date(final)

    n_reg <- switch(tipo,
                    Horarios = {
                        tt <- difftime(final, inicio, units = 'days')
                        tt <- (as.numeric(tt)+1)*48
                        tt <- tt*n_est
                        tt
                    },
                    Diarios = {
                        tt <- difftime(final, inicio, units = 'days')
                        tt <- as.numeric(tt)+1
                        tt <- tt*n_est
                        tt
                    },
                    Semanales = {
                        tt <- difftime(final, inicio, units = 'weeks')
                        tt <- as.numeric(tt)
                        tt <- tt*n_est
                        tt
                    },
                    Mensuales = {
                        tt <- difftime(final, inicio, units = 'weeks')
                        tt <- as.numeric(tt)/4.34524
                        tt <- ceiling(tt)
                        tt <- tt*n_est
                        tt
                    })
    if(n_reg > 100) stop(paste('Number of requested records (', n_reg,
                                ') exceeds the maximum allowed (100)', sep = ''))
    ## Obtain the nearest stations
    siar <- est_SIAR[
        Fecha_Instalacion <= final & (is.na(Fecha_Baja) | Fecha_Baja >= inicio)
    ]

    ## Weigths for the interpolation
    siar[, dist := haversine(Latitud, Longitud, Lat, Lon)]
    siar <- siar[order(dist)][1:n_est]
    siar[, peso := 1/dist]
    siar[, peso := peso/sum(peso)]
    ## Is the given location within the polygon formed by the stations?
    siar <- siar[, .(Estacion, Codigo, dist, peso)]

    ## List for the data.tables of siarGET
    siar_list <- list()
    for(codigo in siar$Codigo){
        siar_list[[codigo]] <- siarGET(id = codigo,
                                       inicio = as.character(inicio),
                                       final = as.character(final),
                                       tipo = tipo)
        siar_list[[codigo]]$peso <- siar[Codigo == codigo, peso]
    }

    ## Bind the data.tables
    s_comb <- rbindlist(siar_list, use.names = TRUE, fill = TRUE)

    nms <- names(s_comb)
    nms <- nms[-c(1, length(nms))]

    ## Interpole
    res <- s_comb[, lapply(.SD * peso, sum, na.rm = TRUE),
                  .SDcols = nms,
                  by = Dates]

    ## Source
    mainURL <- "https://servicio.mapama.gob.es"
    Estaciones <- siar[, paste(Estacion, '(', Codigo, ')', sep = '')]
    Estaciones <- paste(Estaciones, collapse = ', ')
    source <- paste(mainURL, '\n  -Estaciones:', Estaciones, sep = ' ')

    res <- switch(tipo,
                  Horarios = {dt2Meteo(res, lat = Lat, source = mainURL, type = 'bdI')},
                  Diarios = {dt2Meteo(res, lat = Lat, source = mainURL, type = 'bd')},
                  Semanales = {res},
                  Mensuales = {dt2Meteo(res, lat = Lat, source = source, type = 'prom')})
    return(res)
}
\end{lstlisting}

\section{Clases}
\label{sec:org4753fc3}
\subsection{Sol}
\label{sec:org5b0db8f}
\begin{lstlisting}[numbers=left,language=r,label= ,caption= ,captionpos=b]
setClass(
         Class='Sol', ##Solar angles
         slots = c(
             lat='numeric',#latitud in degrees, >0 if North
             solD='data.table',#daily angles
             solI='data.table',#intradaily angles
             sample='character',#sample of time
             method='character'#method used for geometry calculations
         ),
    validity=function(object) {return(TRUE)}
)
\end{lstlisting}
\subsection{Meteo}
\label{sec:orgc6a6f07}
\begin{lstlisting}[numbers=left,language=r,label= ,caption= ,captionpos=b]
setClass(
    Class = 'Meteo', ##radiation and temperature data
    slots = c(
        latm='numeric',#latitud in degrees, >0 if North
        data='data.table',#data, incluying G (Wh/m2) and Ta
        type='character',#choose between 'prom', 'bd' and 'bdI'
        source='character'#origin of the data
    ),
    validity=function(object) {return(TRUE)}
)
\end{lstlisting}
\subsection{G0}
\label{sec:org58c821c}
\begin{lstlisting}[numbers=left,language=r,label= ,caption= ,captionpos=b]
setClass(
    Class = 'G0',
    slots = c(
        G0D = 'data.table',  #result of fCompD
        G0dm = 'data.table', #monthly means
        G0y = 'data.table',  #yearly values
        G0I = 'data.table',  #result of fCompI
        Ta = 'data.table'    #Ambient temperature
    ),
    contains = c('Sol', 'Meteo'),
    validity = function(object) {return(TRUE)}
)

\end{lstlisting}
\subsection{Gef}
\label{sec:org9e86d6d}
\begin{lstlisting}[numbers=left,language=r,label= ,caption= ,captionpos=b]
setClass(
         Class='Gef',
         slots = c(
           GefD='data.table',  #daily values
           Gefdm='data.table', #monthly means
           Gefy='data.table',  #yearly values
           GefI='data.table',  #result of fInclin
           Theta='data.table', #result of fTheta
           iS='numeric',       #dirt index
           alb='numeric',      #albedo
           modeTrk='character',   #tracking mode
           modeShd='character',   #shadow mode
           angGen='list',         #includes alpha, beta and betaLim
           struct='list',         #structure dimensions
           distances='data.frame' #distances between structures
           ),
         contains='G0',
         validity=function(object) {return(TRUE)}
)
\end{lstlisting}
\subsection{ProdGCPV}
\label{sec:org6bf8255}
\begin{lstlisting}[numbers=left,language=r,label= ,caption= ,captionpos=b]
setClass(
         Class='ProdGCPV',
         slots = c(
           prodD='data.table',  #daily values
           prodDm='data.table', #monthly means
           prody='data.table',  #yearly values
           prodI='data.table',  #results of fProd
           module='list',       #module characteristics
           generator='list',    #generator characteristics
           inverter='list',     #inverter characteristics
           effSys='list'        #efficiency values of the system
           ),
         contains='Gef',
         validity=function(object) {return(TRUE)}
)
\end{lstlisting}
\subsection{ProdPVPS}
\label{sec:org43eea68}
\begin{lstlisting}[numbers=left,language=r,label= ,caption= ,captionpos=b]
setClass(
         Class='ProdPVPS',
         slots = c(
           prodD='data.table',  #daily values
           prodDm='data.table', #monthly means
           prody='data.table',  #yearly values
           prodI='data.table',  #results of fPump
           Pg='numeric',        #generator power
           H='numeric',         #manometric head
           pump='list',         #parameters of the pump
           converter='list',    #inverter characteristics
           effSys='list'        #efficiency values of the system
           ),
         contains='Gef',
         validity=function(object) {return(TRUE)}
)
\end{lstlisting}
\subsection{Shade}
\label{sec:org3f884f3}
\begin{lstlisting}[numbers=left,language=r,label= ,caption= ,captionpos=b]
setClass(
         Class='Shade',
         slots = c(
           FS='numeric',  #shadows factor values
           GRR='numeric', #Ground Requirement Ratio
           Yf='numeric',  #final productivity
           FS.loess='loess', #local fitting of FS with loess
           Yf.loess='loess', #local fitting of Yf with loess
           modeShd='character', #mode of shadow
           struct='list',       #dimensions of the structures
           distances='data.frame', #distances between structures
           res='numeric'           #difference between the different steps of the calculations
           ),
         contains='ProdGCPV',##Resultado de prodGCPV sin sombras (Prod0)
         validity=function(object) {return(TRUE)}
)
\end{lstlisting}
