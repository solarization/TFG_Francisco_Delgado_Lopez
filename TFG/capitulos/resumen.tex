\chapter*{Resumen}
El presente proyecto se enfoca en el desarrollo de un paquete de software estadístico en R, denominado \texttt{solaR2}, diseñado para estimar la productividad de sistemas fotovoltaicos a partir de datos de irradiación solar. Este paquete ofrece herramientas avanzadas para investigaciones reproducibles en el campo de la energía solar fotovoltaica, permitiendo tanto la simulación del rendimiento de sistemas conectados a la red como de sistemas de bombeo de agua alimentados por energía solar. \texttt{solaR2} incluye una serie de clases, métodos y funciones que abarcan desde el cálculo de la geometría solar y la radiación solar incidente en un generador fotovoltaico hasta la estimación precisa de la productividad final de estos sistemas, desde la irradiación global horizontal diaria e intradía.

El diseño modular y basado en clases \texttt{S4} facilita el manejo de series temporales multivariantes y ofrece métodos de visualización avanzados para el análisis del rendimiento en plantas fotovoltaicas a gran escala. Una característica distintiva de \texttt{solaR2} es su implementación apoyada en el paquete \texttt{data.table}, que optimiza la manipulación de grandes volúmenes de datos, permitiendo un procesamiento más rápido y eficiente de las series temporales. Esto es fundamental para un análisis detallado y continuo de los datos solares.

Entre sus funcionalidades más destacadas se encuentran el cálculo de la radiación solar en diferentes planos, la estimación del rendimiento de sistemas fotovoltaicos conectados a la red y de sistemas de bombeo, así como la evaluación y optimización de sombras en los sistemas. Además, el paquete incluye herramientas avanzadas para la visualización estadística del rendimiento, permitiendo analizar tanto series temporales como realizar análisis espaciales en combinación con otros paquetes de R. \texttt{solaR2} es particularmente útil para investigadores y profesionales involucrados en el diseño, evaluación y optimización de sistemas fotovoltaicos, proporcionando un análisis detallado de su rendimiento bajo diversas condiciones de irradiación y temperatura, lo que es esencial para maximizar la eficiencia energética y la rentabilidad de las instalaciones solares.

Además, el paquete es compatible con otras bibliotecas de \texttt{R} para la manipulación de series temporales y la visualización de datos, lo que garantiza la precisión en los cálculos temporales y la integración con datos geoespaciales. En resumen, la creación de \texttt{solaR2} representa una contribución significativa al campo de la energía fotovoltaica, proporcionando una herramienta flexible, reproducible y de fácil uso para el análisis y simulación de sistemas solares. Este TFG no solo detalla el desarrollo técnico del paquete, sino que también presenta aplicaciones prácticas y estudios de caso que demuestran su utilidad en escenarios reales, subrayando su capacidad para mejorar la productividad y eficiencia de los sistemas fotovoltaicos mediante un análisis exhaustivo de la radiación solar y las condiciones ambientales.

\paragraph{Palabras clave:}
geometría solar, radiación solar, energía solar, fotovoltaica, métodos de visualización, series temporales, datos espacio-temporales, S4

\chapter*{Abstract}
This project focuses on the development of a statistical software package in R, named \texttt{solaR2}, designed to estimate the productivity of photovoltaic systems based on solar irradiation data. This package offers advanced tools for reproducible research in the field of photovoltaic solar energy, allowing both the simulation of the performance of grid-connected systems and water pumping systems powered by solar energy. \texttt{solaR2} includes a series of classes, methods, and functions that cover everything from the calculation of solar geometry and the solar radiation incident on a photovoltaic generator to the precise estimation of the final productivity of these systems, from daily and intraday global horizontal irradiation.

The modular and class-based \texttt{S4} design facilitates the handling of multivariate time series and offers advanced visualization methods for performance analysis in large-scale photovoltaic plants. A distinctive feature of \texttt{solaR2} is its implementation supported by the \texttt{data.table} package, which optimizes the handling of large volumes of data, allowing faster and more efficient processing of time series. This is essential for detailed and continuous analysis of solar data.

Among its most notable functionalities are the calculation of solar radiation on different planes, the estimation of the performance of grid-connected photovoltaic systems and pumping systems, as well as the evaluation and optimization of shading in the systems. Additionally, the package includes advanced tools for statistical performance visualization, allowing the analysis of both time series and spatial analysis in combination with other R packages. \texttt{solaR2} is particularly useful for researchers and professionals involved in the design, evaluation, and optimization of photovoltaic systems, providing a detailed analysis of their performance under various irradiation and temperature conditions, which is essential to maximize energy efficiency and the profitability of solar installations.

Furthermore, the package is compatible with other \texttt{R} libraries for time series manipulation and data visualization, ensuring accuracy in temporal calculations and integration with geospatial data. In summary, the creation of \texttt{solaR2} represents a significant contribution to the field of photovoltaic energy, providing a flexible, reproducible, and easy-to-use tool for the analysis and simulation of solar systems. This final degree project not only details the technical development of the package but also presents practical applications and case studies that demonstrate its usefulness in real scenarios, highlighting its ability to improve the productivity and efficiency of photovoltaic systems through comprehensive analysis of solar radiation and environmental conditions.

\paragraph{Keywords:}
solar geometry, solar radiation, solar energy, photovoltaic, visualization methods, time series, spatiotemporal data, S4
