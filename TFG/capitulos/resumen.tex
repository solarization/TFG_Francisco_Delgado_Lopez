\chapter*{Resumen}
El presente proyecto se enfoca en el desarrollo de un paquete de software estadístico en R, denominado solaR2, diseñado para estimar la productividad de sistemas fotovoltaicos a partir de datos de irradiación solar. solaR2 es una evolución del paquete solaR, con mejoras significativas en modularidad y eficiencia. A diferencia de solaR, que utilizaba el paquete zoo para la gestión de series temporales, solaR2 se basa en data.table, lo que optimiza la manipulación de grandes volúmenes de datos y acelera el procesamiento. Este avance es crucial para un análisis más detallado y eficiente en el campo de la energía solar fotovoltaica.

solaR2 ofrece herramientas avanzadas para simular el rendimiento de sistemas conectados a la red y sistemas de bombeo de agua con energía solar. Incluye clases, métodos y funciones que permiten calcular la geometría solar y la radiación solar incidente, así como estimar la productividad final de estos sistemas a partir de la irradiación global horizontal diaria e intradía.

El diseño modular basado en clases S4 facilita el manejo de series temporales multivariantes y proporciona métodos de visualización avanzados para el análisis de rendimiento en plantas fotovoltaicas a gran escala. La implementación con data.table mejora la eficiencia en la manipulación de datos, permitiendo análisis más rápidos y precisos. Entre sus funcionalidades destacadas están el cálculo de radiación solar en diferentes planos, la estimación de rendimiento de sistemas fotovoltaicos y de bombeo, y la evaluación de sombras.

Además, solaR2 ofrece herramientas avanzadas para la visualización estadística del rendimiento, compatible con otros paquetes de R para manipulación de series temporales y análisis espacial. Esto la convierte en una herramienta útil para investigadores y profesionales en el diseño y optimización de sistemas fotovoltaicos, permitiendo un análisis detallado bajo diversas condiciones. En resumen, solaR2 representa una mejora significativa en el análisis y simulación de sistemas solares, proporcionando una herramienta flexible y reproducible para mejorar la eficiencia energética y la rentabilidad de las instalaciones solares.

\paragraph{Palabras clave:}
geometría solar, radiación solar, energía solar, fotovoltaica, métodos de visualización, series temporales, datos espacio-temporales, S4

\chapter*{Abstract}
This project focuses on the development of a statistical software package in R, called solaR2, designed to estimate the productivity of photovoltaic systems based on solar irradiation data. solaR2 represents an evolution of the existing solaR package, featuring significant improvements in modularity and efficiency. Unlike solaR, which relied on the zoo package for time series management, solaR2 uses data.table, optimizing the handling of large data volumes and speeding up processing. This advancement is crucial for more detailed and efficient analysis in the field of photovoltaic solar energy.

solaR2 provides advanced tools for simulating the performance of grid-connected systems and solar-powered water pumping systems. It includes classes, methods, and functions for calculating solar geometry and incident solar radiation, as well as estimating the final productivity of these systems from global horizontal irradiation on a daily and intraday basis.

The modular design based on S4 classes facilitates the management of multivariate time series and provides advanced visualization methods for performance analysis in large-scale photovoltaic plants. The use of data.table enhances data handling efficiency, allowing for faster and more precise analyses. Key functionalities include calculating solar radiation on different planes, estimating the performance of photovoltaic and pumping systems, and evaluating shading.

Additionally, solaR2 offers advanced statistical visualization tools, compatible with other R packages for time series manipulation and spatial analysis. This makes it a valuable tool for researchers and professionals involved in the design and optimization of photovoltaic systems, enabling detailed analysis under various conditions. In summary, solaR2 represents a significant improvement in the analysis and simulation of solar systems, providing a flexible and reproducible tool to enhance energy efficiency and the profitability of solar installations.

\paragraph{Keywords:}
solar geometry, solar radiation, solar energy, photovoltaic, visualization methods, time series, spatiotemporal data, S4
